\documentstyle[12pt,fullpage]{article}

\title{PAMI -- User Manual}
\author{Mark Wolforth, Greg Ward}
\date{Tuesday July 6, 1993}  

\def\code#1{{\tt #1}}

\begin{document} 

\maketitle
\newpage

\tableofcontents

%--------------------------------------------------------------
\newpage
\section{Introduction}

This document is the user manual for PAMI (Package for the Analysis of
Medical Images) developed at the Montreal Neurological Institute.
This package consists of a set of Matlab .m files and programs written
in C, designed to ease the analysis of PET data.  The entire package
runs under Matlab.

This document presumes that the reader is familiar with manipulation
of PET images, and simply wishes to know how image manipulation is
possible using PAMI and Matlab.


%--------------------------------------------------------------
\newpage
\section{Basic Concepts}

To a programmer, the interface to PAMI should be very familiar.
Simply open a data set contained in a MINC file by calling openimage.
This returns a handle which is used in all subsequent calls to PAMI
functions.  The user does not need to worry about the underlying
mechanisms.  For example, to open an image file, read in the the first
frame of the second slice, and then close the image file, you could
type:


\begin{verbatim}
data_handle = openimage ('arnaud_20547.mnc');
image = getimages (data_handle, 2, 1);
closeimage (data_handle);
\end{verbatim}


In this example, \verb+data_handle+ is a handle to the data contained in the
MINC file \verb+arnaud_20547.mnc+.  After the data has been opened with
openimage, the name of the MINC file is never used again.  Only the handle for
the data is used.  Therefore, \verb+getnextline+ is passed the handle for the
data, as well as the slice number and frame number.  Once you have finished
working with the image, always remember to call \verb+closeimage+.  This
cleans up the Matlab memory space and destroys the handle.

Because PAMI uses handles, the user never needs to worry about how the image
data is stored within Matlab.  You simply call PAMI functions with the handle,
and the underlying structure takes care of the nitty-gritty details.


%--------------------------------------------------------------
\newpage
\section{PAMI Matlab Functions}

The Matlab interface to PAMI is performed through the following
Matlab scripts and functions (.m files).


\begin{itemize}

\item \code{openimage} : Prepares a MINC file for reading.  Returns a handle to the data.
\begin{verbatim}
IMhandle = openimage (filename)
\end{verbatim}

\item \code{getnextline} : Gets the next line from a data set.  Will get the next line from a
    group of images.
\begin{verbatim}
line = getnextline (IMhandle, slices, frames)
\end{verbatim}

\item \code{getimage} : Gets an entire image from a data set.
\begin{verbatim}
im = getimage (IMhandle, slice, frame)
\end{verbatim}

\item \code{newimage} : Creates a MINC file for a new image.  Returns a handle
    to the newly created data set that is used in calls to other functions
    (such as \code{putnextline} and \code{closeimage}).  Requires a parent
    MINC file from which to copy the history attribute, and root variable.
\begin{verbatim}
IMhandle = newimage (filename, parent_file, numslices, numframes)
\end{verbatim}

\item \code{putnextline} :

\item \code{putimage} :

\item \code{getimageinfo} : Gets information about an open image.  Will return
    one of:
    \begin{itemize}
    \item \code{Filename}  : The name of the MINC file.
    \item \code{NumFrames} : Number of frames in the study. 0 if non-dynamic study.
    \item \code{NumSlices} : Number of slices in the study.
    \item \code{ImageSize} : Size of a single image (eg. 128, 256)
    \item \code{FrameLengths} : Vector with NumFrames elements containing the duration
          of each frame in the study, expressed in seconds.
    \item \code{FrameTimes} : Vector with NumFrames elements containing the start
          time of each frame, relative to the start of the study, expressed
          in seconds.
    \end{itemize}
\begin{verbatim}
info = getimageinfo (IMhandle, info_descriptor)
\end{verbatim}

\item \code{viewimage} : Displays an image on the workstation screen.  If the
    workstation is monochrome, the image will be converted to a grayscale and
    displayed using xv.  Otherwise, the image is displayed in colour using
    Matlab directly.
\begin{verbatim}
viewimage (im)
\end{verbatim}

\item \code{closeimage} : Destroys the appropriate variables in the workspace.
\begin{verbatim}
closeimage (IMhandle)
\end{verbatim}

\end{itemize}


%--------------------------------------------------------------
\newpage
\section{Underlying Structure}

PAMI has a set of underlying functions written in C that provide the interface
to MINC files.  A PAMI user will not ordinarily require use of these functions
since every interaction with MINC files should be possible using the Matlab
scripts and functions provided (.m files).  However, for the sake of
completeness, these functions are documented in this section.

Matlab provides an interface to external C programs that allows dynamic linking
at runtime.  This interface, called CMEX, is detailed in the Matlab
{\em External Interface Guide}.  Please consult this guide for further details.
Unfortunately, due to problems with the CMEX interface, it was not possible to
create any CMEX programs that wrote to MINC files.  Therefore, all of the
MINC creation and writing functions are written as stand-alone C programs, and
all of the MINC reading functions are written as CMEX programs.


\subsection {Stand-alone C Programs}

\begin {itemize}

\item \code{miwriteimages} : Writes images to a MINC file.  The images are taken from
    the specified temporary file, whose data is expected to
    be stored as doubles.  MINC file must exist, and contain
    an image structure (create with micreate and then
    micreateimage).
\begin{verbatim}
int miwriteimages <MINC file> <slices> <frames> <temp file>
\end{verbatim}

\item \code{micreateimage} : Create the MIimage variable in an existing MINC file.
    Also updates the file hierarchy (MIimage is a child of MIrootvariable).
\begin{verbatim}
int micreateimage <MINC file> <size> <number of slices> <number of frames>
\end{verbatim}

\item \code{miwritevar} : Writes values to a variable in a MINC file.  Values are
    taken from the specified temporary file, whose data is
    expected to be stored as doubles.
\begin{verbatim}
int miwritevar <MINC file> <var name> <start vector> <length vector> <temp file>
\end{verbatim}

\item \code{micreatevar} : Create a variable in a MINC file.
\begin{verbatim}
int micreatevar <MINC file> <var name> <data type> <num dims> <dim names>
\end{verbatim}

\item \code{micreatedim} : Create a dimension in a MINC file.
\begin{verbatim}
int micreatedim <MINC file> <dim name> <length>
\end{verbatim}


\item \code{micreate} : Create a new MINC file by copying the global attributes
    and root variable from another MINC file.  Will also copy 
    the specified variables, and update the file hierarchy.  Adds 
    a time stamp to the "history" global variable.
\begin{verbatim}
micreate <parent file> <new file> [<variable list>]
\end{verbatim}

\end{itemize}


\subsection{CMEX Programs}

In addition to stand-alone C programs, the underlying structure also includes
several CMEX programs which are dynamically linked by Matlab.  Please see the
Matlab {\em External Interface Guide} for an explanation of CMEX files.

\begin{itemize}

\item \code{mireadvar} : Read a hyperslab from a MINC (or any NetCDF) file into a MATLAB
    vector, using NetCDF-style start and count vectors.  (Zero-based!!!)  
\begin{verbatim}
mireadvar ('minc_file', 'var_name', start_vector, count_vector)
\end{verbatim}

\item \code{mireadimages} : Read images from a MINC file.
\begin{verbatim}
mireadimages ('minc_file'[, slices[, frames[, options]]]) 
\end{verbatim}

\end{itemize}



\end{document}

