\documentstyle[11pt,fullpage]{article}

\title{EMMA: Extensible MATLAB Medical Analysis \\ User Manual}

\author{Mark Wolforth \and Greg Ward \and Sean Marrett}
\date{20 November, 1995}

\def\code#1{{\tt \bf #1}}

\begin{document} 

\maketitle
\newpage

\tableofcontents

%--------------------------------------------------------------
\newpage
\section{Introduction}


This document is the user manual for EMMA (Extensible MATLAB Medical
Analysis) developed at the Montreal Neurological Institute.  This
package consists of a set of MATLAB .m files and programs written in
C, designed to ease the analysis of medical image data.  The entire
package runs under MATLAB 4.X.

This document assumes that the reader is familiar with manipulation of
medical images (specifically images stored in the MINC file
format)\footnote{MINC and the associated netCDF package are available
via anonymous FTP from ftp.bic.mni.mcgill.ca, in \code{/pub/minc}.},
and simply wishes to know how image manipulation is possible using EMMA
and MATLAB.

The authors would like to thank Sean Marrett of the McConnell Brain
Imaging Centre for being the driving force behind this project.  He
provided invaluable guidance when all seemed lost.

%--------------------------------------------------------------
\newpage
\section{Basic Concepts}

To a programmer, the interface to EMMA should be very familiar.
Simply open a data set contained in a MINC file by calling the
\code{openimage} function.  This returns a handle which is used in all
subsequent calls to EMMA functions.  The user does not need to worry
about the underlying mechanisms.  For example, to open an image file,
read in the the first frame of the second slice, and then close the
image file, you could type:

%
\begin{verbatim}
     data_handle = openimage ('arnaud_20547.mnc');
     image = getimages (data_handle, 2, 1); 
     closeimage (data_handle);
\end{verbatim}
%

In this example, \verb+data_handle+ is a handle to the data contained
in the MINC file \verb+arnaud_20547.mnc+.  After the data file has
been opened with \code{openimage}, the name of the MINC file is not
needed by the user of EMMA; only the handle for the data is used.
Therefore, \code{getimages} is passed the handle for the data, as well
as the slice number and frame number.  Once you have finished working
with the image, always remember to call \code{closeimage}.  This
cleans up global variables created in the MATLAB workspace.

Because EMMA uses handles, the user never needs to worry about the
interface between MATLAB and the MINC file.  You simply call EMMA
functions with the handle, and the underlying structure takes care of
the details.

However, the user {\em does} need to have some familiarity with
MATLAB, and a good understanding of how EMMA functions deal with
images.  In particular, the inherent two-dimensionality of MATLAB
means that a full four-dimensional MINC file is ``flattened'' in some
way when read into MATLAB.

Generally, EMMA functions deal with images transformed from their
natural two-dimensional form (well-suited to a matrix representation)
to a one-dimensional vector form.  For example, a single 128 $\times$
128 image will become a {\em column} vector of dimension 16384
$\times$ 1.  We can put several such vectors together in a matrix.
For example, we can form a 16384 $\times$ 15 matrix from 15 slices.
Thus, in an ``image matrix''---which is a two-dimensional MATLAB
matrix representing three-dimensional image data---the individual
columns correspond to whole images.  MATLAB functions that use EMMA to
perform analysis should always expect image data in this format, as
shown in Section
\ref{simple_example} and the document ``rCBF Analysis Using MATLAB.''

Also, keep in mind that EMMA makes it deliberately difficult to deal
simultaneously with the full four dimensions allowed in a MINC image
variable.  The reasoning for this is two-fold: first, collapsing three
dimensions into two via the ``image matrix'' concept is fairly
convenient and easy to work with; however, attempting to similarly
collapse four dimensions would add another layer of complications,
making both the implementation and use of EMMA functions more
difficult.  Second is the question of memory usage: MATLAB is capable
of using up enormous amounts of memory if care is not taken, and
restricting the user of EMMA to three image dimensions at a time means
that it is somewhat more difficult to blindly read in massive amounts
of image data with a single MATLAB command.

%--------------------------------------------------------------
\newpage
\section{Basic EMMA MATLAB Functions}

The basic MATLAB interface to EMMA is performed through the following
MATLAB functions (.m files).  This section is only meant to provide a
brief introduction to the most important functions.  See section
\ref{emma_reference} for full help on every EMMA function.

\begin{description}

\item \code{openimage} - Prepares a MINC file for reading.
\code{openimage} determines the image size, reads in the frame start
times and lengths if applicable, and returns a handle which can be
used to access the MINC file with \code{getimages},
\code{getimageinfo}, \code{getblooddata}, etc.
\begin{verbatim}
     handle = openimage (filename)
\end{verbatim}

\item \code{getimages} - Gets images from a MINC file.  If the file
has no frame dimension, the parameter \verb|frames| should be empty
or not provided.  For files with no slices (only frames), the
\verb|slices| parameter should be made empty (a matrix of the form
\verb|[]|).  Also, \code{getimages} can read several frames from a
single slice or several slices from a single frame---but it can not
read multiple slices and multiple frames simultaneously.  (This
restriction is also imposed for writing images, mainly to avoid
complications and reduce the software's ability to easily read in far
more data than can be handled by MATLAB.)

Two important features of \code{getimages} are the ability to re-use
memory, and the ability to read partial images.  Both of these
features are intended to reduce the amount of memory used by MATLAB
when processing images.
\begin{verbatim}
     Images = getimages (handle [, slices [, frames [, old_matrix ...
                         [, start_row [, num_rows]]]]])
\end{verbatim}

\item \code{newimage} - Creates a MINC file for a new set of images.
Returns a handle to the newly created data set that is used in calls
to other functions (such as \code{putimages} and \code{closeimage}).
\code{newimage} has a number of optional parameters, the most
important of which is the ``parent file.''  If this is supplied,
\code{newimage} will copy a number of important attributes (such as
the patient, study, and acquisition data) from the parent to the new
file, as well as using the parent file to provide the default image
size and type.
\begin{verbatim}
     handle = newimage (filename, dim_sizes [, parent_file [, image_type ...
                        [, valid_range [, orientation]]]]);
\end{verbatim}

\item \code{putimages} - Writes entire images to a file created with
\code{newimage}.  ({\bf Warning:} there are currently no provisions
made to deny the user from writing to a MINC file opened with
\code{openimage}.)  The syntax is similar to \code{getimages}, except
that the image data is of course an input argument to \code{putimages}
rather than an output argument.
\begin{verbatim}
     putimages (handle, images [, slices [, frames]])
\end{verbatim}

\item \code{getimageinfo} - Gets information about an open image.
Currently, \code{getimageinfo} will return the filename, number of
frames or slices, image height, width, or size, the dimension sizes,
frame start times, lengths, and mid-frame times.  The desired
information is specified as a character string, eg. 'NumFrames',
'ImageHeight', etc.
\begin{verbatim}
     info = getimageinfo (handle, info_descriptor)
\end{verbatim}

\item \code{viewimage} - Displays an image on the workstation screen.
On a monochrome display, the \code{gray} colourmap will be used;
otherwise, the \code{spectral} will be used.  The image will be scaled
such that the highest point in the image is always the last colour in
the colourmap, and the lowest image point will be the first element of
the colourmap.  Also, a colourbar relating colours in the image to the
values will be displayed, unless the optional \code{colourbar\_flag}
argument is set to zero.

\begin{verbatim}
     viewimage (image [, colourbar_flag])
\end{verbatim}

\item \code{closeimage} - Destroys the appropriate variables in the workspace.
\begin{verbatim}
     closeimage (IMhandle)
\end{verbatim}

\end{description}

\section{Some Basic Examples: Masking and Summing}
\label{simple_example}

As an example of the EMMA functions presented thus far, we will
explore sample code for both masking and summing/integrating images
from a dynamic study.  The MINC file used in this example will be one
of the H$_2$$^{15}$O CBF studies from
\code{/usr/local/matlab/toolbox/emma/examples}; there is also a
corresponding BNC ({\bf b}lood {\bf n}et{\bf C}DF) file, which is
necessary to get the plasma activity data.  To make accessing
these files easier, you may want to create symbolic links to them in a
directory of your own:
\begin{verbatim}
     cd <your-work-directory>
     ln -s /local/matlab/toolbox/emma/examples/yates_19445.* .
\end{verbatim}
That way, the file
\verb|/local/matlab/toolbox/emma/examples/yates_19445.mnc| can be
accessed as simply \verb|yates_19445.mnc| (as long as you are in your
work directory).  In the remainder of this section, it will be assumed
that these symbolic links exist in the current directory.

\subsection{Reading the Data}

First, let us open the MINC file:
\begin{verbatim}
     handle = openimage ('yates_19445.mnc');
\end{verbatim}
Now, we may wish to find out some basic data about the MINC file.  Try
the following (note that we will need \verb|nf|, the number of frames,
below):
\begin{verbatim}
     ns = getimageinfo (handle, 'NumSlices')
     nf = getimageinfo (handle, 'NumFrames')
     getimageinfo (handle, 'ImageSize')
\end{verbatim}
Note that for the study \verb|yates_19445|, \verb|nf| will be 21.
However, it is a good idea to always use \verb|getimageinfo| to find
the number of frames rather than assuming that it will always stay the
same for a particular type of study.

Now, let us read in the frame lengths and mid-frame times:
\begin{verbatim}
     FrameLengths = getimageinfo (handle, 'FrameLengths');
     MidFTimes = getimageinfo (handle, 'MidFrameTimes');
\end{verbatim}
Note that \verb|FrameLengths| and \verb|MidFTimes| are both column
vectors.  This will be important when performing matrix arithmetic
with them.

Now, we read in all frames for one slice---slice 8 in this case,
although you can work with whatever slice you like:
\begin{verbatim}
     slice = 8;
     PET = getimages (handle, slice, 1:nf);
\end{verbatim}
Note that {\em this} is the step where MATLAB uses up large amounts of
memory.  If you execute the MATLAB \verb|whos| command now, you will
see that MATLAB's variables alone are taking up nearly 3 megabytes of
memory.  However, the entire MATLAB workspace generally takes up
considerably more memory than is reported by \verb|whos|, because of
MATLAB's overhead and inefficient memory management scheme.

\subsection{Integrating Images}

We can view the matrix \verb|PET| as a collection of 21 PET images
for the same brain slice but progressing in time, or as a collection
of 16,384 time-activity curves (TAC)---one for every pixel in the 128
$\times$ 128 image.  (Of course, only about $\frac{1}{3}$ of the
image data is actually data from inside the head.  This matter will
be addressed when we consider image masking below.)

For purposes of integrating images, we will consider \verb|PET| as a
collection of TACs.  Thus, we are simply performing 16384 numerical
integrations.  This may seem a daunting task, but MATLAB's vectorized
approach to mathematics makes it surprisingly easy.  In fact, let us
calculate a rectangular integration using the formula:
\begin{equation}
\int_{t_{1}}^{t_{n}} f(t) dt \approx \sum_{i=1}^{n} f(t_{i}) (\Delta t)_{i},
\label{eq:rect_int}
\end{equation}
where $f(t)$ is one time-activity curve and the $\Delta t$'s are the
frame lengths.  Since we have $f(t)$ stored as a row
vector---remember, each column of \verb|PET| is the entire image for
one frame, so each row of \verb|PET| corresponds to a single pixel
across time---and $\Delta t$ stored as a column vector, the
multiply/sum operation of equation (\ref{eq:rect_int}) is simply a
dot product.  And since a matrix multiplication is nothing more than
a sequence of dot products, all 16,384 TACs can be quite easily
integrated with a single MATLAB command:
\begin{verbatim}
     PETsummed1 = PET * FrameLengths;
\end{verbatim}

However, rectangular integration is a very crude approach to
numerical integration.  Trapezoidal integration provides a slightly
more refined method, and is almost as easy to implement.
Furthermore, it is provided by the MATLAB function
\verb|trapz|---which, if you care to examine it (try \verb|type
trapz| in MATLAB), is essentially a one-line operation as well.
(Incidentally, one of the functions provided by EMMA is
\code{ntrapz}.  \code{ntrapz} currently has exactly the same
restrictions and capabilities as \verb|trapz|, but is considerably
faster due to being written in C using MATLAB's C interface.  For the
purposes of this discussion, the two functions are equivalent.)

\verb|trapz| takes two arguments: a vector of $x$-points, and a vector
or matrix of $y$ points corresponding to each $x$-point.  If the $y$'s
are given as a matrix, then each {\em row} corresponds to one $x$
point.  Note that this is different from the EMMA standard, where each
{\em column} of \verb|PET| corresponds to one time point.  Thus, we
will have to transpose \verb|PET| using the MATLAB \verb|'| operator
before passing it to \verb|trapz|.  Also, \verb|trapz| will return the
16,384 integrals as a row vector, so we must transpose the results to
get them back into the form we expect.  Finally, \verb|trapz| expects
points on the $x$-axis rather than widths of intervals.  Thus, we will
pass \verb|MidFTimes| to it, and not \verb|FrameLengths|.  The full
integration is performed by the single statement
\begin{verbatim}
     PETsummed2 = trapz (MidFTimes, PET')';
\end{verbatim}

{\bf Note:} This method of calculating the integral is
extremely wasteful of memory due to transposing the PET images.  For a
full discussion of MATLAB memory issues, please consult the online
document Controlling MATLAB Memory
Use\footnote{http://www.mni.mcgill.ca/system/mni/matlab/matlab\_memory.html}.



\subsection{Viewing and Comparing the Integrated Images}

Now, we wish to view the integrated images.  If you are using a colour
console or an monochrome X terminal, enter the following to create a
MATLAB figure window and display the image obtained by rectangular
integration:
\begin{verbatim}
     figure;
     viewimage (PETsummed1);
\end{verbatim}
(Note that explicitly creating the window via the \verb|figure|
command is not really necessary.  However, if you already had a
figure window with a plot or image that you did not want to lose,
MATLAB would have overwritten the existing figure when you called
\code{viewimage}.  It is generally good practice to call
\verb|figure| before creating new plots or images, unless you are
sure you wish to obliterate whatever was previously in the current
figure window.)

You may want to title this window:
\begin{verbatim}
     title ('yates_19445, slice 8, rectangular integration');
\end{verbatim}

Now do the same thing for \verb|PETsummed2|.  You will probably want
to move one or both figure windows so that you can see both images
simultaneously.  Clearly, the two images are similar---but if we want
a more objective comparison, we can calculate the ratio of the two
images on a pixel-by-pixel basis and view this:
\begin{verbatim}
     ratio = PETsummed1 ./ PETsummed2;
     figure;
     viewimage (ratio);
\end{verbatim}

Note the use of the \verb|./| operator to specify an
element-by-element operation on two matrices.  The resulting image
shows that most of the pixels in \verb|PETsummed1| and
\verb|PETsummed2| appear fairly close.  However, because some pixels
have such large values compared to 1, any small variations around 1
are swamped.  One way to deal with this is simply to clip the image
at certain points: for example, set all points below -10 to -10, and
all points above 10 to 10.  This is accomplished as follows:
\begin{verbatim}
     clipneg = find (ratio < -10);
     ratio (clipneg) = ones (size (clipneg)) * (-10);
     clippos = find (ratio > 10);
     ratio (clippos) = ones (size (clippos)) * (10);
\end{verbatim}

If you wish to know how many points were clipped at either end,
type \verb|size (clipneg)| or \verb|size (clippos)|.

If you now \code{viewimage (ratio)}, it should be clear that the
overall ratio between the two integrated images is close to 1 inside
the head.

\subsection{Saving the Integrated Image}

After performing some analysis, it is often desirable to write the
results to a new MINC file.  Currently, we only have a single image to
write, so we will create a MINC file with 1 slice and no frames
(because of the integration across frames, time is no longer a
variable).  We use EMMA function \code{newimage} as follows:
\begin{verbatim}
     new_handle = newimage ('yates_summed.mnc', [0 1], ...
                            'examples/yates_19445.mnc');
\end{verbatim}
To write our single image to slice 1 of the new MINC file, we enter:
\begin{verbatim}
     putimages (new_handle, PETsummed2, 1);
\end{verbatim}

Alternately, we could create a MINC file with a full 15 slices, and
loop through the entire original study to sum every slice.  MATLAB
will allow us to enter a \verb|for| loop interactively, so enter the
following (note that if you already executed the above code, you
should use a new filename in place of \code{yates\_summed.mnc} or
delete the old one---\code{newimage} will die if you try to overwrite
an existing file):
\begin{verbatim}
     closeimage (new_handle);
     new_handle = newimage ('yates_summed.mnc', [0 ns], ...
                            'examples/yates_19445.mnc');
     for slice = 1:ns
        PET = getimages (handle, slice, 1:nf);
        PETsummed = trapz (MidFTimes, PET')';
        putimages (new_handle, PETsummed, slice);
     end
     closeimage (new_handle);
\end{verbatim}

\subsection{Using an Integrated Image for Image Masking}

First, let us clear some of the unneeded variables from the workspace:
\begin{verbatim}
     clear PETsummed1 ratio clipneg clippos
\end{verbatim}

Now, we will use \verb|PETsummed2| for a very simple form of image
masking: namely, we will use only pixels whose values in the
integrated image are greater than the mean of the integrated image.
For CBF studies, this usually creates a usable mask that contains only
in-brain voxels.  It can be done by calculating the mean of the
integrated image, and creating another image consisting entirely of
1's and 0's (a ``binary'' image), where a 1 indicates that the pixel
is inside the head, and a zero indicates outside:
\begin{verbatim}
     avg = mean (PETsummed2);
     mask = PETsummed2 > avg;
\end{verbatim}

We can view the mask as an image itself to see a silhouette of the
brain at this slice:
\begin{verbatim}
     figure;
     viewimage (mask);
\end{verbatim}
Or, we can apply the mask to the summed image and view it.  To apply
the mask to an image, simply perform an element-by-element multiply 
using the
\verb|.*| operator on the mask and the image; since \verb|mask|
consists entirely of ones and zeros, every point in the image will
either be set to zero or unchanged.  Thus,
\begin{verbatim}
     PETsummed2 = PETsummed2 .* mask;
     figure;
     viewimage (PETsummed2);
\end{verbatim}
will display the trapezoidally integrated image with out-of-head data
set to zero.

Finally, to see just how many pixels are ``on'' in the mask, you can type
\begin{verbatim}
     size (find (mask))
\end{verbatim}
The fact that only a fraction of the entire image consists of image
data means that masking can be used to great advantage when
performing analysis that requires computations for every pixel.
(Generally, this should be avoided if it is all possible to take
advantage of MATLAB's vectorized structure.)

%--------------------------------------------------------------
\newpage
\section{Underlying Structure}

EMMA has a set of underlying functions written in C that provide the
interface to MINC files.  An EMMA user will not ordinarily require
use of these functions since every interaction with MINC files should
be possible using the MATLAB scripts and functions provided (.m
files).  However, for the sake of completeness, these functions are
documented in this section.

MATLAB provides an interface to external C programs that allows
dynamic linking at runtime.  This interface, called CMEX, is detailed
in the MATLAB {\em External Interface Guide}.  Please consult this
guide for further details.  Unfortunately, due to problems with the
CMEX interface, it was not possible to create any CMEX programs that
wrote to MINC files.  Therefore, all of the MINC creation and writing
functions are written as stand-alone C programs, and all of the MINC
reading functions are written as CMEX programs.


\subsection {Stand-alone C Programs}

The following programs are used by various EMMA functions and are not
normally needed by the user.

\begin {itemize}

\item \code{miwriteimages} - Writes images to a MINC file.  The
images are taken from the specified temporary file, whose data is
expected to be stored as doubles.  The MINC file must exist, and
contain an image structure (create with micreate and then
micreateimage).
\begin{verbatim}
    miwriteimages <MINC file> <slices> <frames> <temp file>
\end{verbatim}

\item \code{micreateimage} - Create a new MINC file, complete with
image dimensions, dimension and dimension-width variables, image
max/min variables, and (of course) the image variable.  If the new
file is based on a ``parent'' MINC file, any other variables and
global attributes will be copied from the parent to the new file.  As
much information as {\em can} be copied about the image and image
dimensions will also be copied, but the fact that the new file may or
may not have the same orientation, dimensions, and dimension lengths
as its parent complicates matters greatly.
\begin{verbatim}
    micreateimage <MINC file> [option] [option] ...
\end{verbatim}

\item \code{miwritevar} - Writes values to a variable in a MINC file.  
Values are taken from the specified temporary file, whose data is
expected to be stored as doubles.
\begin{verbatim}
    miwritevar <MINC file> <var name> <start vector> ...
               <lengthvector> <temp file>
\end{verbatim}

\end{itemize}


\subsection{CMEX Programs}

In addition to stand-alone C programs, the underlying structure also includes
several CMEX programs which are dynamically linked by MATLAB.  Please see the
MATLAB {\em External Interface Guide} for an explanation of CMEX files.

\begin{itemize}

\item \code{mireadvar} - Read a hyperslab from a MINC (or any NetCDF)
file into a MATLAB vector, using NetCDF-style start and count
vectors.  Note that the indexing of \verb|start_vector| and
\verb|stop_vector| is zero-based.
\begin{verbatim}
    mireadvar ('minc_file', 'var_name', start_vector, count_vector)
\end{verbatim}

\item \code{mireadimages} - Read images from a MINC file.  Opens the
given MINC file, and attempts to read whole or partial images from the
slices and frames specified in the slices and frames vectors.  If
\verb|start_row| and \verb|num_rows| are not specified, then whole
images are read.  If only \verb|start_row| is specified, a single row
will be read.  If both \verb|start_rows| and \verb|num_rows| are
given, then the specified number of rows will be read (unless either
is out of bounds, which will result in an error message).

\begin{verbatim}
    mireadimages ('minc_file'[, slices[, frames[, options]]]) 
\end{verbatim}

\item \code{miinquire} - find out various things about a MINC file
from MATLAB.  Retrieves some item(s) of information about a MINC file.
The first argument is always the name of the MINC file.  Following the
filename can come any number of ``option sequences'', which consist of
the option (a string) followed by zero or more items (more strings).
Generally speaking, the ``option'' tells miinquire the general class
of information you're looking for (such as an attribute value or a
dimension length), and the item or items that follow it give
\code{miinquire} more details, such as the name of a dimension,
variable, or attribute.

\begin{verbatim}
    miinquire ('minc_file' [, 'option' [, 'item']], ...)
\end{verbatim}

\end{itemize}

%--------------------------------------
\newpage
\subsection{MATLAB files}
\label{emma_reference}

\def\endfunchelp{\par\medskip\noindent\rule{\textwidth}{1pt}\par\bigskip}

{\large\bf CALPIX} {\em  generates the vector index of a point}
\begin{verbatim}


        cp = calpix(x,y)


\end{verbatim}

  generates the vector index (1 to 16384) of 
  the x and y coordinates of a pixel in a 128*128 image.
  It uses the simple formula 
 
        cp = round(x) + 128 * round(y-1)
 
\endfunchelp

%--------------------------------------

{\large\bf CHECK\_SF} {\em  for internal use only}
\begin{verbatim}


       msg = check_sf (handle, slices, frames)


\end{verbatim}

\begin{itemize}
\item \code{msg = []}: all is OK
\item \code{msg} = error message if there's anything wrong
\end{itemize}

  makes sure that the given slices and frames vectors
  are consistent with the image specified by handle.  Checks for:

\begin{itemize}
\item both slices and frames cannot be vectors
\item if image has no frames, frames vector should be empty
\item if frames vector is empty, image must have no frames
\item if image has no slices, slices vector should be empty
\item if slices vector is empty, image must have no slices
\end{itemize}

\endfunchelp

%--------------------------------------

{\large\bf CLOSEIMAGE} {\em  closes image data set(s)}
\begin{verbatim}


      closeimage (handles)


\end{verbatim}

  Closes one or more image data sets.
\endfunchelp

%--------------------------------------

{\large\bf DERIV} {\em  calculate the derivative and smoothed version of a function}
\begin{verbatim}


       [yfit, deriv] = deriv (fit_points, data_points, y, dt)


\end{verbatim}

   This function calculates the derivative and smoothed
   version of a function, using the method of parabolic
   regressive filters, described in Sayers: "Inferring
   Significance from Biological Signals."
 
\endfunchelp

%--------------------------------------

{\large\bf GETBLOODDATA} {\em retrieve blood activity and sample times from a study}
\begin{verbatim}


         [activity, mid_times] = getblooddata (study)


\end{verbatim}

   The study variable can be a handle to an open image, or the name of 
   a NetCDF (MNC or BNC) file containing the blood activity data for
   the study.  If it is a handle, getblooddata will first look in the
   associated MINC file (if any), and then in the associated BNC file
   (if any) for the blood activity variables.  If just a filename
   (either a MINC or BNC file) is given, getblooddata will look in that
   file only.
  
   The mid-sample times will be calculated from amongst the variables
   sample\_start, sample\_stop, and sample\_length.  The default is
   mid = (sample\_start + sample\_stop)/2; but if sample\_stop is not
   found in the MNC or BNC file, then mid = sample\_start + (sample\_length/2)
   will be used instead.
 
   If a file that does not exist is specified, or getblooddata cannot
   find the blood activity data in either the MINC or BNC file, it will
   print a warning message and return nothing (ie. empty matrices).
\endfunchelp

%--------------------------------------

{\large\bf GETIMAGEINFO} {\em   retrieve helpful trivia about an open image}
\begin{verbatim}


      info = getimageinfo (handle, whatinfo)


\end{verbatim}

  Get some information about an open image.  handle refers to a MINC
  file previously opened with openimage or created with newimage.
  whatinfo is a string that describes what you want to know about.
  The possible values of this string are numerous and ever-expanding.
  
  The first possibility is the name of one of the standard MINC image
  dimensions: 'time', 'zspace', 'yspace', or 'xspace'.  If these are
  supplied, getimageinfo will return the length of that dimension from
  the MINC file, or 0 if the dimension does not exist.  Note that
  requesting 'time' is equivalent to requesting 'NumFrames'; also,
  the three spatial dimensions also have equivalences that are
  somewhat more complicated.  For the case of transverse images,
  zspace is equivalent to NumSlices, yspace to ImageHeight, and xspace
  to ImageWidth.  See the help for newimage (or the MINC standard
  documentation) for details on the relationship between image
  orientation (transverse, sagittal, or coronal) and the MINC spatial
  image dimensions.
  
  The other possibilities for whatinfo, and what they cause
  getimageinfo to return, are as follows:
 
\begin{description}
\item \code{Filename}     - the name of the MINC file (if applicable)
                     as supplied to openimage or newimage; will be
                     empty if data set has no associated MINC file.
\end{description}
 
\begin{description}
\item \code{NumFrames}    - number of frames in the study, 0 if non-dynamic
                     study (equivalent to 'time')
\end{description}
 
\begin{description}
\item \code{NumSlices}    - number of slices in the study (0 if no slice
                     dimension)
\end{description}
 
\begin{description}
\item \code{ImageHeight}  - the size of the second-fastest varying spatial 
                     dimension in the MINC file.  For transverse
                     images, this is just the length of MIyspace.
                     Also, when an image is displayed with viewimage,
                     the dimension that is "vertical" on your display
                     is the image height dimension.  (Assuming
                     viewimage is working correctly.)
\end{description}
 
\begin{description}
\item \code{ImageWidth}   - the size of the fastest varying spatial
                     dimension, which is MIxspace for transverse
                     images.  When an image is displayed with
                     viewimage, the image width is the horizontal
                     dimension on your display.
\end{description}
 
\begin{description}
\item \code{ImageSize}    - a two-element vector containing ImageHeight and
                     ImageWidth (in that order).  Useful for viewing 
                     non-square images, because viewimage needs to know
                     the image size in that case.
\end{description}
 
\begin{description}
\item \code{DimSizes}     - a four-element vector containing NumFrames, NumSlices,
                     ImageHeight, and ImageWidth (in that order)
\end{description}
 
\begin{description}
\item \code{FrameLengths} - vector with NumFrames elements - duration of
                     each frame in the study, in seconds.  This is
                     simply the contents of the MINC variable
                     'time-width'; if this variable does not exist in
                     the MINC file, then getimageinfo will return an
                     empty matrix.
\end{description}
 
\begin{description}
\item \code{FrameTimes}   - vector with NumFrames elements - start time of
                     each frame, relative to start of study, in
                     seconds.  This comes from the MINC variable
                     'time'; again, if this variable is not found,
                     then getimageinfo will return an empty matrix.
\end{description}
 
\begin{description}
\item \code{MidFrameTimes} - time at the middle of each frame (calculated by
                      FrameTimes + FrameLengths/2) in seconds
\end{description}
       
  If the requested data item is invalid or the image specified by handle
  is not found (ie. has not been opened), then the returned data will
  be an empty matrix.  (You can test whether this is the case with
  the isempty() function.)
 
  SEE ALSO  openimage, newimage, getimages
\endfunchelp

%--------------------------------------
{\large\bf GETIMAGES} {\em  Retrieve whole or partial images from an open MINC file.}
\begin{verbatim}


   images = getimages (handle [, slices [, frames [, old_matrix ...
                       [, start_row [, num_rows]]]]])


\end{verbatim}

   reads whole or partial images from the MINC file specified by
   handle.  Either slices or frames can be a vector (to specify a
   set of several images), but at least one of them must be a scalar
   -- it is not possible to read images from both different slices and
   different frames at the same time.  (Multiple calls to getimages
   will be needed for this.)  If the file is non-dynamic (no time
   dimension), then the frames argument can be omitted or empty;
   likewise, if there is no slice dimension, the slices argument can
   be omitted or empty.  (But note that slices must be given if any
   frames are to be specified -- thus, it may be necessary to supply
   an empty matrix for slices in the unusual case of a MINC file with
   frames but no slice variation.)
 
   The default behaviour of getimages is to read whole images and
   return them as MATLAB column vectors with the image rows stored
   sequentially.  If multiple images are read, then they will be
   returned as the columns of a matrix.  For instance, if 10 128x128
   images are read, then getimages will return a 16384x10 matrix; to
   extract a single image, use MATLAB's colon operator, as in foo
   (:,1) to extract all rows of column 1 of the matrix foo.
 
   To read partial images, you can specify a starting image row in
   start\_row; if num\_rows is not supplied and start\_row is, then a
   single row is read.  
 
   To try to conserve memory use, you can "recycle" MATLAB matrices
   when sequentially calling getimages to read in identically-sized
   blocks of image data.  This is done by simply passing your image
   matrix to getimages as old\_matrix, eg:
 
   img = [];
   for slice = 1:numslices
      img = getimages (handle, slice, 1:numframes, img);
      (process img)
   end
 
   This will get around MATLAB's tendency to unnecessarily allocate
   new blocks of memory and leave old blocks unused.
 
   EXAMPLES (assuming handle = openimage ('some\_minc\_file');)
 
    To read in the first frame of the first slice:
      one\_image = getimages (handle, 1, 1);
    To read in the first 10 frames of the first slice:
      first\_10 = getimages (handle, 1, 1:10);
    To read in the first 10 slices of a non-dynamic (i.e. no frames) file:
      first\_10 = getimages (handle, 1:10);
    
   Note that there is currently no way to write partial images -- this 
   feature is provided in the hopes of cutting down memory usage due
   to intermediate calculations; you should pre-allocate a matrix large
   enough to hold your final results, and place them there as blocks of 
   rows from the input MINC file are processed.  Then, when all rows
   have been processed, a whole output image can be written to the
   output file.
\endfunchelp

%--------------------------------------


{\large\bf GETMASK} {\em returns a mask that is the same size as the passed image.}
\begin{verbatim}


      mask = getmask (image)


\end{verbatim}

  The mask consists of 0's and 1's, and is created interactively by
  the user.  Currently, a threshold algorithm is used, based on the input
  argument image: the user selects a threshold using a slider (the default
  starting value is 1.8), and getmask selects all points in image greater
  than the mean value of the entire image multiplied by threshold the
  threshold.  It then displays image as masked by that threshold value, so
  the user can refine the threshold to his/her satisfaction.
\endfunchelp

%--------------------------------------

{\large\bf GETPIXEL} {\em replacement for MATLAB's ginput function}
\begin{verbatim}


      [x,y] = getpixel(n)


\end{verbatim}

  MATLAB's ginput function crashes if there is no X display
  defined.  This function checks to make sure that the display
  exists before calling ginput.  The functionality of this
  function is exactly the same as MATLAB's ginput.
\endfunchelp

%--------------------------------------

{\large\bf HOTMETAL} {\em a better hot metal color map.}
\begin{verbatim}


          map = hotmetal(num_colors)


\end{verbatim}

  HOTMETAL(M) returns an M-by-3 matrix containing a "hot" colormap.
  HOTMETAL, by itself, is the same length as the current colormap.
 
  For example, to reset the colormap of the current figure:

\begin{verbatim}
 
            colormap(hotmetal)
 
\end{verbatim}

  See also HSV, GRAY, PINK, HOT, COOL, BONE, COPPER, FLAG,
           COLORMAP, RGBPLOT, SPECTRAL.
\endfunchelp

%--------------------------------------

{\large\bf MAKETAC} {\em Make a time-activity curve}
\begin{verbatim}


      tac = maketac(x,y,pet)


\end{verbatim}

      Generate a time-activity curve from a set of data.
\endfunchelp

%--------------------------------------

{\large\bf MIINQUIRE} {\em   find out various things about a MINC file from MATLAB}
\begin{verbatim}


    info = miinquire ('minc_file' [, 'option' [, 'item']], ...)


\end{verbatim}

  miinquire has a rather involved syntax, so pay attention.  The first
  argument is always the name of a MINC file.  Following the filename
  can come any number of "option sequences", which consist of the option
  (a string) followed by zero or more items (more strings).
 
  Any number of option sequences can be included in a single call to 
  miinquire, as long as enough output arguments are provided (this is
  checked mainly as a debugging aid to the user).  Generally, each option
  results in a single output argument.
 
  The currently available options are:
 
      dimlength
      imagesize
      vartype
  
  Options that will most likely be added in the near future are:
 
      dimnames
      varnames
      vardims
      varatts
      atttype
      attvalue
 
  One inconsistence with the standalone utility mincinfo (after which 
  miinquire is modelled) is the absence of the option "varvalues".  
  The functionality of this available in a superior way via the CMEX
  mireadvar.
 
  EXAMPLES
 
   [ImSize, NumFrames, ImType] = ...
     miinquire ('foobar.mnc', 'imagesize', ...
                'dimlength', 'time', ...
                'vartype', 'image');
 
  puts the four-element vector of image dimension sizes into ImSize; 
  the length of the time dimension into the scalar NumFrames; and the
  type of the image variable into the string ImType.
\endfunchelp

%--------------------------------------

{\large\bf MIREADIMAGES} {\em  Read images from specified slice(s)/frame(s) of a MINC file.}
\begin{verbatim}


   images = mireadimages ('minc_file'[, slices[, frames[, options]]])


\end{verbatim}

   opens the given MINC file, and attempts to read whole images from
   the slices and frames specified in the slices and frames vectors.
   For the case of 128 x 128 images, the images are returned as the 
   the columns of a 16384-row matrix, with the highest image dimension
   varying the fastest.  That is, if x is the highest image dimension,
   each contiguous block of 128 elements will correspond to one row
   of the image.
 
   To manipulate a single image as a 128x128 matrix, it is necessary
   to extract the desired column (image), and then reshape it to the
   appropriate size.  For example, to load all frames of slice 5, and 
   then extract frame 7 of the file foobar.mnc:

\begin{verbatim} 
   >> images = mireadimages ('foobar.mnc', 4, 0:20);
   >> frame7 = images (:, 7);
   >> frame7 = reshape (frame7, 128, 128);
\end{verbatim}
 
   Note that mireadimages expects slice and frame numbers to be zero-
   based!  Thus, frames 0 .. 20 of the MINC file are read into columns
   1 .. 21 of the matrix images.
 
   For most dynamic analyses, it will also be necessary to extract
   the frame timing data.  This can be done using MIREADVAR.
 
   Currently, only one of the vectors slices or frames can contain multiple
   elements.
\endfunchelp

%--------------------------------------

{\large\bf MIREADVAR} {\em  Read a hyperslab of data from any variable in a MINC file.}
\begin{verbatim}


   data = mireadvars ('MINC_file', 'var_name', [, start, count[, options]])


\end{verbatim}

   Given vectors describing the starting corner (zero-based!) and edge
   lengths, mireadvars reads an n-dimensional hyperslab from a MINC
   (or NetCDF) file.  The data is returned as a MATLAB vector, with
   the highest dimension of the variable changing fastest.
  
   The simplest (and intended) use of mireadvars is to read an entire
   one-dimensional variable.  For example:
 
     time = mireadvars ('foobar.mnc', 'time');
 
   will read the entire contents of the variable 'time' from the file
   foobar.mnc.  (If the start and count vectors are not given, they
   default to reading the entire variable.  Currently, if start is
   given, count must be given, and they must each have exactly one
   element per dimension.)
 
   A more complicated example is to use mireadvar as a low-rent
   substitute for mireadimages.  For example, to read slice 5, frame 7
   (note that these are 1-based, and mireadvar expects 0-based
   indeces!) of foobar.mnc:
 
     image = mireadvar ('foobar.mnc', 'image', [6 4 0 0], [1 1 128 128]);
 
   The disadvantages of this approach are numerous.  First of all,
   mireadimages will perform the scaling and shifting necessary to
   transform the image data from scaled bytes or shorts (or however it
   happens to be stored in the MINC file) to floating point values
   representing the actual physical data.  Second, with mireadvar you
   must know the exact order of the dimensions: in the above example,
   "slice 5" corresponds to the 4 (note zero-based!) at position 2 of
   the Start vector, and "frame 7" is the 6.  Also, you must know the
   size of the image; mireadimages will handle anything, not just
   128x128.  Finally, mireadimages provides greater flexibility with
   respect to slice and frame selection.  With mireadvar, you can only
   read contiguous ranges of slices and frames, and you must figure
   out the start and count values for each dimension yourself.
   Mireadimages, however, does all that work for you given just slice
   and frame numbers.
\endfunchelp

%--------------------------------------

{\large\bf MIWRITEIMAGES} {\em  write images to a MINC file}
\begin{verbatim}


    miwriteimages (filename, images, slices, frames)


\end{verbatim}

   writes images (in the format as returned by mireadimages) to a MINC
   file.  The MINC file must already exist and must have room for the 
   data.  slices and frames only tell miwriteimages where to put the data
   in the MINC file, they are not used to select certain columns from images.
 
   Also, the slices and frames must be valid and consistent with the MINC
   file, which must exist and have an image variable in it.  The number
   of images to write (implied by the number of elements in slices or frames)
   must be the same as the number of columns in the matrix images.  Since
   miwriteimages only expects to be called by putimages, none of these
   requirements are checked here -- all that is done by putimages.
 
   Note that there is also a standalone executable miwriteimages; this 
   is called by miwriteimages.m via a shell escape.  Neither of these
   programs are meant for everyday use by the end user.
\endfunchelp

%--------------------------------------

{\large\bf NCONV} {\em  Convolution of two vectors with not necessarily unit spacing.}
\begin{verbatim}


 	C = NCONV(A, B, spacing) convolves vectors A and B.  The resulting
 	vector is length LENGTH(A)+LENGTH(B)-1.


\end{verbatim}

        This routine is a replacement for MathWorks' conv function, 
        which implicitly assumes that A and B are sampled with unit
        spacing.  If you are dealing with two functions that are
        unevenly sampled or sampled with different spacings, one or
 	both of them must be resampled to the same evenly spaced 
 	independent variable.  Then, if the spacing of the independent
 	variable is not 1, it should be passed to nconv.
 
 	See also CONV, XCORR, DECONV, CONV2, LOOKUP.
\endfunchelp

%--------------------------------------

{\large\bf NEWIMAGE} {\em  create a new MINC file, possibly descended from an old one}
\begin{verbatim}


   handle = newimage (NewFile, DimSizes, ParentFile, ...
                      ImageType, ValidRange, DimOrder)


\end{verbatim}

  creates a new MINC file.  NewFile and DimSizes must always be given,
  although the number of elements required in DimSizes varies
  depending on whether ParentFile is given (see below).  All other
  parameter are optional, and, if they are not included or are
  empty, default to values sensible for PET studies at the MNI.
 
  The optional arguments are:
 
\begin{description}
\item \code{ParentFile} - the name of an already existing MINC file.  If this
                 is given, then a number of items are inherited from
                 the parent file and included in the new file; note
                 that this can possibly change the defaults of all
                 following optional arguments.
\end{description}
 
\begin{description}
\item \code{DimSizes}   - a vector containing the lengths of the image
                 dimensions.  If ParentFile is not given, then all
                 four image dimensions (in the order frames, slices,
                 height, and width) must be specified.  Either or both
                 of frames and slices may be zero, in which case the
                 corresponding MINC dimension (MItime for frames, and
                 one of MIzspace, MIyspace, or MIxspace for slices)
                 will not be created.  If ParentFile is given, then
                 only the number of frames and slices are required; if
                 the height and width are not given, they will default
                 to the height/width of the parent MINC file.  In no
                 case can the height or width be zero -- these two
                 dimensions must always exist in a MINC file.  See
                 below, under "DimOrder", for details on how slices,
                 width, and height are mapped to MIzspace, MIyspace,
                 and MIxspace for the various conventional image
                 viewpoints.
\end{description}
 
\begin{description}
\item \code{ImageType}  - a string, containing a C-like type dictating how the
                 image is to be stored.  Currently, this may be one of
                 'byte', 'short', 'long', 'float', or 'double'; plans
                 are afoot to add 'signed' and 'unsigned' options for
                 the three integer types.  Currently, 'byte' images will
                 be unsigned and 'short' and 'long' images will be
                 signed.  If this option is empty or not supplied, it
                 will default to 'byte'.  NOTE: this parameter is currently
                 ignored.
\end{description}
                 
\begin{description}
\item \code{ValidRange} - a two-element vector describing the range of possible 
                 values (which of course depends on ImageType).  If
                 not provided, ValidRange defaults to the maximum
                 range of ImageType, eg. [0 255] for byte, [-32768
                 32767] for short, etc.  NOTE: this parameter is currently
                 ignored.
\end{description}
 
\begin{description}
\item \code{DimOrder}   - a string describing the orientation of the images,
                 one of 'transverse' (the default), 'sagittal', or
                 'coronal'.  Transverse images are the default if
                 DimOrder is not supplied.  Recall that in the MINC
                 standard, zspace, yspace, and xspace all have
                 definite meanings with respect to the patient: z
                 increases from inferior to superior, x from left to
                 right, and y from posterior to anterior.  However,
                 the concepts of slices, width, and height are
                 relative to a set of images, and the three possible 
                 image orientations each define a mapping from
                 slices/width/height to zspace/yspace/xspace as
                 follows:
\end{description}
 
                     Orientation  Slice dim    Height dim   Width dim
                      transverse   MIzspace     MIyspace     MIxspace
                      sagittal     MIxspace     MIzspace     MIyspace
                      coronal      MIyspace     MIzspace     MIxspace
\endfunchelp

%--------------------------------------

{\large\bf NFRAMEINT} {\em   integrate a function across a range of intervals (frames)}
\begin{verbatim}


    integrals = nframeint (ts, y, FrameStarts, FrameLengths)


\end{verbatim}

  calculates the integrals of a function (represented as a set of points
  in y, sampled at the time points in ts) across each of a set of frames
  which are given by their start times and lengths.  The integral is then
  normalised so that nframeint returns the average value of y (as a function
  of ts) across each frame.
 
  ts and y must be vectors of the same length, as must FrameStarts and 
  FrameLengths.  Normally, ts and y are a good deal longer than
  FrameStarts and FrameLengths in order to get reasonably accurate 
  results.  The returned variable, integrals, will be a vector of
  the same length of FrameStarts and FrameLengths, containing the 
  integral of y(ts) across each frame.
 
  Points of y to integrate for each frame are selected by finding all
  points of ts that are greater than the frame start time and less than
  the frame stop time.  If possible, y is then linearly interpolated at
  the frame start and stop times to form a closed interval.  Then, a
  trapezoidal integration across those points is calculated, and the
  integral is divided by the width of the interval across which y
  is known within the frame.  Normally, this will simply be the length
  of the frame.  However, it may be that the lowest value of ts is
  greater than the frame start or the highest value of ts is lower than
  the frame stop time.  In these cases, y is not known outside of the
  frame, and cannot be resampled at the frame endpoints; so, the integration
  will only be performed across known points, and the integral will
  be divided not by the length of the frame but by the width of the interval
  across which y is known.
\endfunchelp

%--------------------------------------

{\large\bf OPENIMAGE} {\em   setup appropriate variables in MATLAB for reading a MINC file}
\begin{verbatim}


   handle = openimage (filename)


\end{verbatim}

   Sets up a MINC file and prepares for reading.  This function
   creates all variables required by subsequent get/put functions such
   as getimages and putimages.  It also reads in various data about
   the size and number of images on the file, all of which can be
   queried via getimageinfo.
 
   The value returned by openimage is a handle to be passed to getimages,
   putimages, getimageinfo, etc.
\endfunchelp

%--------------------------------------

{\large\bf PUTIMAGES} {\em  Writes whole images to an open MINC file.}
\begin{verbatim}


       putimages (handle, images [, slices [, frames]])


\end{verbatim}

   writes images (a matrix with each column containing a whole image)
   to the MINC file specified by handle, at the slices/frames specfied
   by the slices/frames vectors.
 
   Note that only one of the vectors slices or frames may have multiple
   elements; ie., you may not write multiple slices and multiple frames
   simultaneously.  (This should not be a problem, since you cannot *read*
   multiple frames and slices simultaneously either.)  If both slices
   and frames are present in the MINC file, then both slices and frames
   vectors must be supplied and be non-empty.  If either of those 
   dimensions are not present, though, then the associated vector must
   be either omitted or empty.  
 
   EXAMPLES
     
     To write zeros to an entire slice (say, 21 frames of slice 7) of
     a full dynamic MINC file with 128x128 images [already opened with
     handle = newimage (...)]:
 
       images = zeros (16384,21);
       putimages (handle, images, 7, 1:21);
 
     To write random data to a single slice (7) of a non-dynamic file 
     (again 128x128 images):
 
       image = rand (16384, 1);
       putimages (handle, image, 7);
 
   SEE ALSO  newimage, openimage, getimages
\endfunchelp

%--------------------------------------

{\large\bf RESAMPLEBLOOD} {\em  resample the blood activity in some new time domain}
\begin{verbatim}


   [new_g, new_ts] = resampleblood (handle, type[, samples])


\end{verbatim}

   reads the blood activity and sample timing data from the study
   specified by handle, and resamples the activity data at times
   specified by the string type.  Currently, type can be one of 'even'
   or 'frame'.  For 'even', a new, evenly-spaced set of times will be
   generated and used as the resampling times.  For 'frame', the mid
   frame times will be used.  In either case, the resampled blood
   activity is returned as new\_g, and the times used are returned as
   new\_ts.
 
   The optional argument samples specifies the number of samples
   to take.  If it is not supplied, resampleblood will resample the
   blood data at roughly 0.5 second intervals.
\endfunchelp

%--------------------------------------

{\large\bf SMOOTH} {\em  do a simple spatial smoothing on an image}
\begin{verbatim}


         new_data = smooth (old_data)


\end{verbatim}

  Smooths a two-dimensional image by averaging over a circle with a
  diameter of 5 pixels.
\endfunchelp

%--------------------------------------

{\large\bf SPECTRAL} {\em Black-purple-blue-green-yellow-red-white color map.}
\begin{verbatim}


          map = spectral(num_colors)


\end{verbatim}

  SPECTRAL(M) returns an M-by-3 matrix containing a "spectral" colormap.
  SPECTRAL, by itself, is the same length as the current colormap.
 
  For example, to reset the colormap of the current figure:
 
            colormap(spectral)
 
  See also HSV, GRAY, PINK, HOT, COOL, BONE, COPPER, FLAG,
           COLORMAP, RGBPLOT.
\endfunchelp

%--------------------------------------

{\large\bf TEMPFILENAME} {\em generate a unique temporary filename}
\begin{verbatim}


          fname = tempfilename


\end{verbatim}

  Requires that a directory /tmp/ exists on the current machine.
\endfunchelp

%--------------------------------------

\endfunchelp

%--------------------------------------

{\large\bf VIEWIMAGE} {\em  displays a PET image from a vector or square matrix.}
\begin{verbatim}


     [fig_handle, image_handle, bar_handle] = viewimage (img, colourbar_flag)


\end{verbatim}

   viewimage (img) sets the colourmap to spectral (the standard PET
   colourmap) and uses MATLAB's image function to display the image.
   Works on either SGI's or Xterminals, with colours dithered
   to black and white on the Xterms.
 
   Images are scaled so its high points are white and its low points black.
 
   viewimage (img, colourbar\_flag) turns the colourbar on or off.
   The default is on, but by specifying colourbar flag = 0, the
   colourbar will be turned off.
 
\endfunchelp

%--------------------------------------



\end{document}

