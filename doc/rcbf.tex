\documentstyle[12pt,fullpage]{article}

\title{rCBF Analysis Using MATLAB}
\author{Mark Wolforth and Greg Ward}

\def\code#1{{\tt #1}}

\begin{document}

\maketitle
\newpage

\tableofcontents

%--------------------------------------------------------------
\newpage
\section{Introduction}


%--------------------------------------------------------------
\newpage
\section{Mathematical Analysis}

\subsection{Introduction}

The two-compartment cerebral blood flow model can be characterized by
the following three equations:

\begin{equation}
\frac{dM}{dt} = K_{1}C_{a}(t) - k_{2}M(t)      \label{eq:2comp1}
\end{equation}
\begin{equation}
M(t) = K_{1}C_{a}(t) \otimes e^{-k_{2}t}       \label{eq:2comp2}
\end{equation}
\begin{equation}
A(t) = M(t) + C_{a}(t)V_{0}                    \label{eq:2comp3}
\end{equation}

In these equations, $A(t)$ is the PET data collected over a set of
frames, $C_{a}(t)$ is the delay and dispersion corrected arterial
blood sample data, and $M(t)$ is the radioactive tracer activity
present in cerebral tissue.  We know both $A(t)$ and $C_{a}(t)$, but
cannot know $M(t)$ without knowing $V_{0}$.

One additional point to consider is that these equations are written
for continuous functions.  However, the PET data that is actually
available is average activity across each frame.  Therefore, in
order to solve the equations, we must average the non PET data
across frames.  By combining equations (\ref{eq:2comp2}) and
(\ref{eq:2comp3}), and averaging the non PET data over frames, we
get:

\begin{equation}
A^{*}(t) = \frac{\int_{T_1}^{T_2} K_{1} \left[ C_{a}(t) \otimes
e^{-k_{2}t} \right] + C_{a}(t)V_{0} dt}{T_{2} - T_{1}}
\label{eq:2comp4}
\end{equation}

where $A^{*}(t)$ is the PET data that is actually collected (by
definition, averaged over frames), and ${\int_{T_1}^{T_2}dt} / {(T_2 -
T_1)}$ performs an averaging over frames ($T_1$ is the frame start
time, $T_2$ is the frame stop time, and the integral is evaluated for
each frame).

We wish to solve equation (\ref{eq:2comp4}) for $K_{1}$, $k_{2}$, and
$V_{0}$.  Of course, one approach would be to try to perform an
explicit least squares curve fitting.  However, this approach would be
quite computationally intensive since the fitting would need to be
performed for every pixel of a 128x128 pixel image.  Fortunately,
there is a method of solution that gives good results with reduced
computational difficulty.

\subsection{Double-Weighted Integration Method}

\label{sec:double_weight}

Since the time required to perform a least squares curve fitting would
be prohibitive, a simpler approach to solving the problem is required.
One technique is to use a weighted integration method.  We initially
approached the problem by assuming that $V_{0}$ was negligibly small,
in which case the $C_{a}V_{0}$ term is eliminated from equation
(\ref{eq:2comp4}).  Taking equation (\ref{eq:2comp4}) with $C_{a}V_0$
eliminated, and integrating both sides from time 0 to the end of the
last frame, we get:

\begin{equation}
\int_{0}^{T} A^{*}(t) dt = K_{1} \int_{0}^{T} \frac{\int_{T_1}^{T_2}
\left[ C_{a}(t) \otimes e^{-k_{2}t} \right] dt}{T_2 - T_1}
\end{equation}

We can then take this equation, and divide it by a weighted version
of itself:

\begin{equation}
\frac{\int_{0}^{T} A^{*}(t) dt}{\int_{0}^{T} A^{*}(t) t dt} =
\frac{K_{1} \int_{0}^{T} \frac{\int_{T_1}^{T_2} \left[ C_{a}(u)
\otimes e^{-k_{2}u} \right] du}{T_2 - T_1} dt}{K_{1} \int_{0}^{T}
\frac{\int_{T_1}^{T_2} \left[ C_{a}(u) \otimes e^{-k_{2}u} \right]
du}{T_2 - T_1} t dt}
\label{eq:1comp}
\end{equation}

$K_{1}$ cancels out of this equation, leaving us with an equation
that only involves $k_{2}$.  The left side of equation
(\ref{eq:1comp}) is easily evaluated by integrating the PET data.
The right side of equation (\ref{eq:1comp}) is not easily solved for
$k_{2}$, so a different approach was taken.  A look-up table was
generated relating values of $k_{2}$ to resulting values of the
right hand side of equation (\ref{eq:1comp}).  A linear
interpolation was then performed to choose values of $k_{2}$ from
this look-up table for each point in the left hand side of equation
(\ref{eq:1comp}).


\subsection{Triple-Weighted Integration Method}

In order to model the complete two-compartment system, we must be able
to solve equation (\ref{eq:2comp4}).  In section
\ref{sec:double_weight},
we solved the equation by multiplying by two different weights and
then dividing, and we can take a similar approach with the full
two-compartment equation.  We can weight equation (\ref{eq:2comp4})
with {\em three} different weights and then integrate:

\begin{equation}
\int_{0}^{T} w_{1}A^{*}(t)dt = K_{1} \int_{0}^{T} w_{1}
\frac{\int_{T_1}^{T_2} \left[ C_{a}(u) \otimes e^{-k_{2}u} \right]
du}{T_2 - T_1} dt + V_{0} \int_{0}^{T}w_{1} \frac{\int_{T_1}^{T_2}
C_{a}(u) du}{T_2 - T_1} dt \label{eq:fullweight1}
\end{equation}

\begin{equation}
\int_{0}^{T} w_{2}A^{*}(t)dt = K_{1} \int_{0}^{T} w_{2}
\frac{\int_{T_1}^{T_2} \left[ C_{a}(u) \otimes e^{-k_{2}u} \right]
du}{T_2 - T_1} dt + V_{0} \int_{0}^{T}w_{2} \frac{\int_{T_1}^{T_2}
C_{a}(u) du}{T_2 - T_1} dt \label{eq:fullweight2}
\end{equation}

\begin{equation}
\int_{0}^{T} w_{3}A^{*}(t)dt = K_{1} \int_{0}^{T} w_{3}
\frac{\int_{T_1}^{T_2} \left[ C_{a}(u) \otimes e^{-k_{2}u} \right]
du}{T_2 - T_1} dt + V_{0} \int_{0}^{T}w_{3} \frac{\int_{T_1}^{T_2}
C_{a}(u) du}{T_2 - T_1} dt \label{eq:fullweight3}
\end{equation}

By multiplying equation (\ref{eq:fullweight1}) by $V_{0}
\int_{0}^{T} w_{3} {(\int_{T_1}^{T_2}C_{a}(u)du)}/{(T_2 - T_1)} dt$
and equation (\ref{eq:fullweight3}) by $V_{0} \int_{0}^{T} w_{1}
{(\int_{T_1}^{T_2}C_{a}(u)du)}/{(T_2 - T_1)} dt$, and then
subtracting the two, we may eliminate the $V_{0}$ term.  A similar
operation can be performed on equation (\ref{eq:fullweight2}) and
equation (\ref{eq:fullweight3}).  This leaves two equations that do
not contain $V_{0}$.  They may then be divided to produce:

%
\begin{equation}
\hspace*{-1.0cm}
\frac{
\begin{array}{lcr}
\multicolumn{2}{l}{
\int_{0}^{T}\!\!w_{3}(t) \frac{(\int_{T_1}^{T_2}C_{a}(u)du)}{(T_2 - T_1)} dt \cdot 
\int_{0}^{T}\!\!w_{1}(t)A^{*}(t)dt\,\,-}&\\
&\multicolumn{2}{r} {\int_{0}^{T}\!\!w_{1}(t)
\frac{(\int_{T_1}^{T_2}C_{a}(u)du)}{(T_2 - T_1)} dt \cdot
\int_{0}^{T}\!\!w_{3}(t)A^{*}(t)dt}\end{array}
}
{
\begin{array}{lcr}
\multicolumn{2}{l}
{\int_{0}^{T}\!\!w_{3}(t) \frac{(\int_{T_1}^{T_2}C_{a}(u)du)}{(T_2 - T_1)} dt \cdot 
\int_{0}^{T}\!\!w_{2}(t)A^{*}(t)dt\,\,-}&\\
&\multicolumn{2}{r}
{\int_{0}^{T}\!\!w_{2}(t) \frac{(\int_{T_1}^{T_2}C_{a}(u)du)}{(T_2 - T_1)} dt \cdot 
\int_{0}^{T}\!\!w_{3}(t)A^{*}(t)dt}\end{array}
}
 =
\frac{
K_{1} \left\{\!\!\!\! \begin{array}{c}
\begin{array}{lcr}\multicolumn{2}{l}
{\int_{0}^{T}\!\!w_{3}(t) \frac{(\int_{T_1}^{T_2}C_{a}(u)du)}{(T_2 -
T_1)} dt \cdot
\int_{0}^{T}\!\!w_{1}(t) \frac{(\int_{T_1}^{T_2}C_{a}(u) \otimes
e^{-k_{2}u} du)}{(T_2 - T_1)}\, dt\,\,-}&
\\ 
&\multicolumn{2}{r} {\int_{0}^{T}\!\!w_{1}(t)
\frac{(\int_{T_1}^{T_2}C_{a}(u)du)}{(T_2 - T_1)} dt \cdot
\int_{0}^{T}\!\!w_{3}(t)  \frac{(\int_{T_1}^{T_2}C_{a}(u) \otimes
e^{-k_{2}u} du)}{(T_2 - T_1)}\, dt}
\end{array} \end{array} \!\!\!\! \right\} }
{
K_{1} \left\{ \!\!\!\! \begin{array}{c}
\begin{array}{lcr} 
\multicolumn{2}{l}{\int_{0}^{T}\!\!w_{3}(t) \frac{(\int_{T_1}^{T_2}C_{a}(u)du)}{(T_2 - T_1)} dt \cdot 
\int_{0}^{T}\!\!w_{2}(t) \frac{(\int_{T_1}^{T_2}C_{a}(u) \otimes
e^{-k_{2}u} du)}{(T_2 - T_1)}\, dt\,\,-} &
\\
&\multicolumn{2}{r}{\int_{0}^{T}\!\!w_{2}(t)
\frac{(\int_{T_1}^{T_2}C_{a}(u)du)}{(T_2 - T_1)} dt \cdot
\int_{0}^{T}\!\!w_{3}(t)  \frac{(\int_{T_1}^{T_2}C_{a}(u) \otimes
e^{-k_{2}u} du)}{(T_2 - T_1)}\, dt} \end{array} \end{array} \!\!\!\!
\right\} }
\label{eq:k1two}
\end{equation}

The $K_{1}$ term cancels out of both the numerator and denominator of
equation (\ref{eq:k1two}), leaving an equation that only involves
$k_{2}$.  As with the equation in section \ref{sec:double_weight},
this is very difficult to solve for $k_{2}$.  Therefore, a look-up
table was again used.  Once the table matching values of $k_{2}$ with
values of the right hand side of equation (\ref{eq:k1two}) has been
created, we may evaluate $k_{2}$ through simple lookup.  With the
$k_{2}$ data computed, finding $K_{1}$ is simply a matter of
evaluating either the numerator or denominator of equation
(\ref{eq:k1two}) without cancelling $K_{1}$.  With both $K_{1}$ and
$k_{2}$ known, we may find $V_{0}$ by evaluating equation
(\ref{eq:2comp4}).

\subsection{Delay and Dispersion Correction}

Since the blood samples used to estimate $C_a(t)$ are taken from the
subject's arm, the activity data gathered from them must be corrected
for delay and dispersion within the body.  In order to perform the
delay correction, we used the technique of Iida, by performing a
least squares fitting of the equation:

\begin{equation}
A^{*}(t) = \frac{\int_{T_1}^{T_2} \alpha \left[ C_a(t) \otimes e^{-\beta
t} \right] + \gamma C_a(t) dt}{T_2 - T_1} \label{eq:delay_correc}
\end{equation}

where $A^{*}(t)$ in this case is the average activity over grey
matter, $\int_{T_1}^{T_2}dt$ represents integration over frames,
$\alpha$, $\beta$, and $\gamma$ are the fitting parameters, and
$C_a(t)$ is the blood data.  The delay itself enters this equation by
generating $C_a(t)$ from:

\begin{equation}
C_a(t) = \bar{g}(t+\delta)
\end{equation}

The function $\bar{g}$ is the result of performing dispersion
correction on the blood sample data.  This is represented by the
equation:

\begin{equation}
\bar{g}(t) = g(t) + \tau \frac{dg}{dt}
\end{equation}

which is an implicit deconvolution of the equation:

\begin{equation}
g(t) = \bar{g}(t) \otimes \left[ \frac{1}{\tau} e^{\frac{-t}{\tau}}
\right]
\end{equation}


%--------------------------------------------------------------
\newpage
\section{MATLAB Implementation}

\subsection{Introduction}

The previous section described the mathematical model used to
represent cerebral blood flow.  This mathematical description was
implemented numerically in MATLAB.

It is written as six different functions.


\subsection{Program Structure}

\subsection{Annotated Program Listings}

\subsubsection{RCBF2}

\begin{enumerate}
\item Check the input arguments.
\begin{verbatim}
% Input argument checking

if (nargin < 2)
   help rcbf2
   error ('Not enough input arguments');
elseif (nargin < 3)
   progress = 0;
elseif (nargin < 4)
   correction = 1;
elseif (nargin > 4)
   help rcbf2
   error('Incorrect number of arguments.');
end

if (length(slice)~=1)
   help rcbf2
   error('<Slice> must be a scalar.');
end
\end{verbatim}

\item Get the images and image information.  This includes all of the
frames for the slice in question, the frame start times, frame
lengths, mid-frame times, and the blood data.
\begin{verbatim}
img = openimage(filename);
FrameTimes = getimageinfo (img, 'FrameTimes');
FrameLengths = getimageinfo (img, 'FrameLengths');
MidFTimes = FrameTimes + (FrameLengths / 2);
[g_even, ts_even] = resampleblood (img, 'even');


PET = getimages (img, slice, 1:length(FrameTimes));
PET = PET * 37 / 1.05;                  % convert to decay / (g_tissue * sec)
\end{verbatim}

\item Create the weighting functions, plus some useful integrals that
are used several times later in the program.  Here we also create a
simple mask that excludes information outside of the brain, and apply
it to our integrals.  This will ensure that areas outside of the
brain will be set to zero.
\begin{verbatim}
w2 = MidFTimes;
w3 = sqrt (MidFTimes);

ImLen = size(PET,1);
PET_int1 = C_trapz (MidFTimes, PET')';
PET_int2 = C_trapz (MidFTimes, PET' .* (w2 * ones(1,ImLen)))';
PET_int3 = C_trapz (MidFTimes, PET' .* (w3 * ones(1,ImLen)))';

% Now use PET_int1 to create a simple mask, and mask all three PET integrals.
% This does a good job of removing the outside-of-head data for CBF studies.

mask = PET_int1 > mean(PET_int1);
PET_int1 = PET_int1 .* mask;
PET_int2 = PET_int2 .* mask;
PET_int3 = PET_int3 .* mask;
\end{verbatim}

\item Correct the blood data for delay and dispersion, and apply the
cross-calibration factor.  The correction will be accomplished by
performing a curve fitting on average activity across grey matter.
Therefore, \code{getmask()} allows the user to select grey matter by
choosing a threshold value.  The average grey matter activity is then
passed to \code{correctblood()}.
\begin{verbatim}
XCAL = 0.11;
g_even = g_even*XCAL*37;              % units are decay / (g_tissue * sec)

if (correction)
   mask = getmask (PET_int1);
   A = (mean (PET (find(mask),:)))';
   [ts_even, Ca_even, delta] = correctblood ...
                       (A, FrameTimes, FrameLengths, g_even, ts_even, progress);
else
   Ca_even = g_even;
end
\end{verbatim}

\item Generate three very useful lookup tables.  These are
$\int_{0}^{T} w (\int_{T_1}^{T_2} Ca(u) \otimes e^{-k_{2}u} du)/(T_2
- T_1) dt$ evaluated for $w_1$, $w_2$, and $w_3$ as the weights
($w$).  These lookup tables are used in all subsequent evaluations of
these three weighted integrals.  The values of $k_2$ used in the
lookup are also chosen here.
\begin{verbatim}
k2_lookup = (-10:0.05:10) / 60;
[conv_int1,conv_int2,conv_int3] = findintconvo (Ca_even,ts_even,k2_lookup,...
      MidFTimes, FrameLengths, 1, w2, w3);
\end{verbatim}

\item Generate some additional useful integrals.  These are used more
than once in the subsequent code, and so are calculated in advance to
try and increase the speed slightly.
\begin{verbatim}
Ca_mft = nframeint (ts_even, Ca_even, FrameTimes, FrameLengths);      

Ca_int1 = C_trapz(MidFTimes, Ca_mft);
Ca_int2 = C_trapz(MidFTimes, (w2 .* Ca_mft));
Ca_int3 = C_trapz(MidFTimes, (w3 .* Ca_mft));
\end{verbatim}

\item Generate the lookup table relating k2 to values of the right
hand side of equation (\ref{eq:k1two}).  We also calculate the left
hand side of equation (\ref{eq:k1two}), which will be used in the
generation of a $k_2$ image.  Since we will be looking up values of
$k_2$ from values of rR (the right hand side of equation
(\ref{eq:k1two})), we sort the table on rR.
\begin{verbatim}
rL = ((Ca_int3 .* PET_int1) - (Ca_int1 .* PET_int3)) ./ ...
     ((Ca_int3 .* PET_int2) - (Ca_int2 .* PET_int3));

rR = ((Ca_int3 * conv_int1) - (Ca_int1 * conv_int3)) ./ ...
      ((Ca_int3 * conv_int2) - (Ca_int2 * conv_int3));

% Now, we must have the k2/rR lookup table in order by rR; however, 
% we also want to keep k2_lookup in the original order.  This
% is because the first lookup uses rL as a lookup into rR to
% find k2 (which requires that rR be monotonic, ie. sorted) whereas
% subsequent lookups all use k2 to find conv_int{1,2,3} -- which 
% requires that k2_lookup be monotonic.  So k2_lookup will be the
% list of k2's in order, and k2_sorted will be the same list, but 
% in order according to the sorted rR.

[rR,sort_order] = sort (rR);
k2_sorted = k2_lookup (sort_order);
\end{verbatim}

\item Generate the $k_2$ image.  This is performed through a simple
table lookup.
\begin{verbatim}
k2 = lookup(rR, k2_sorted, rL);
\end{verbatim}

\item Generate the $K_1$ image by evaluating the numerator of
equation (\ref{eq:k1two}).  All of the time consuming calculations
have already been performed, and we can evaluate the $\int_{0}^{T} w
(\int_{T_1}^{T_2} Ca(u) \otimes e^{-k_{2}u} du)/(T_2 - T_1) dt$ terms
through table lookup.
\begin{verbatim}
K1_numer = ((Ca_int3*PET_int1) - (Ca_int1 * PET_int3));
K1_denom = (Ca_int3 * lookup(k2_lookup,conv_int1,k2)) - ...
           (Ca_int1 * lookup(k2_lookup,conv_int3,k2));
K1 = K1_numer ./ K1_denom;
\end{verbatim}

\item Generate the $V_0$ image by evaluating equation
(\ref{eq:fullweight1}).
\begin{verbatim}
V0 = (PET_int1 - (K1 .* lookup(k2_lookup,conv_int1,k2))) / Ca_int1;
\end{verbatim}

\end{enumerate}






%--------------------------------------------------------------
\newpage
\section{MATLAB Functions}

{\large\bf B\_CURVE} {\em a two-compartment rCBF model used for delay
correction}
\begin{verbatim}


      integral = b_curve (args, shifted_g_even, ts_even, A, ... 
                          fstart, flengths)


\end{verbatim}

  Used by blood delay correction, this function implements a
  two-compartment rCBF model used for fitting the blood
  curve data to the head curve data.
\newpage

%--------------------------------------

{\large\bf CORRECTBLOOD} {\em  perform delay and dispersion corrections on blood curve}
\begin{verbatim}


   [new_ts_even, Ca_even, delta] = correctblood (A, FrameTimes, ...
                                   FrameLengths, g_even, ts_even, progress)


\end{verbatim}

   The required input parameters are: 
\begin{description}
\item \code{A} - brain activity, averaged over all gray matter in a slice.  This
           should be in units of decay / (gram-tissue * sec), and should
           just be a vector - one value per frame.
\item \code{FrameTimes} - the start time of every frame, in seconds
\item \code{FrameLengths} - the length of every frame, in seconds
\item \code{g\_even} - the (uncorrected) arterial input function, resampled at
                some *evenly spaced* time domain.  Should be in units
                of decay / (mL-blood * sec)
\item \code{ts\_even} - the time domain at which g\_even is resampled
\end{description}
 
   The returned variables are:
\begin{description}
\item \code{new\_ts\_even} - generally the same as the old time scale,
                     with some points missing from the end.
\item \code{Ca\_even} - g\_even with dispersion and delay hopefully corrected,
                 in units of decay / (mL-blood * sec).  
\item \code{delay} - the delay time (ie. shift) in seconds
\end{description}
 
   A, FrameTimes, and FrameLengths must all be vectors with the same
   number of elements (presumably the number of frames in the study).
   g\_even and ts\_even must also be vectors with the same number of
   elements, but their size should be much larger, due to the
   resampling at half-second intervals performed by resampleblood.
   
   correctblood corrects for dispersion in blood activity by
   calculating g(t) + tau * dg/dt, where tau (the dispersion time
   constant) is taken to be 4.0 seconds.
 
   It then attempts to correct for delay by fitting a theoretical blood
   curve to the observed brain activity A(t).  This curve depends
   on the parameters alpha, beta, gamma (these correspond to K1, k2,
   and V0, although for the entire slice rather than pixel-by-pixel) and
   delta (which is the delay time).  correctblood steps through a series
   of delta values (currently -5 to +10 sec), and performs a three-
   parameter fit with respect to alpha, beta, and gamma; the value of
   delta that results in the best fit is chosen as the delay time.
 
   options is an entirely optional vector meant for debugging purposes.
   If options(1) is non-zero, then correctblood will show its progress,
   by printing out the results of progressive delay-correction fits.  If 
   it is at least 2, then correctblood will also show progress graphically,
   by displaying a graph of A(t) and the fits corresponding to every
   value of delta tried.  If options(2) is zero, then no delay correction
   will be performed; if options(3) is supplied, then it will be
   used as delta to do delay correction without the time-consuming
   fitting.
\newpage

%--------------------------------------

{\large\bf FINDINTCONVO} {\em   calculate tables of the integrated convolutions commonly used}
\begin{verbatim}


    [int1,int2,int3] = findintconvo (Ca_even, ts_even, k2_lookup,...
                                     midftimes, flengths, w1[, w2[, w3]])


\end{verbatim}

  given a table of k2 values, generates tables of weighted integrals
  that commonly occur in RCBF analysis.  Namely, int\_convo is a table of
  the same size as k2\_lookup containing
 
        int ( conv (Ca(t), exp(-k2*t)) * weight )
 
  where the integration is carried out across frames.  weight is
  one of w1, w2, or w3, each of which will generally be some simple
  function of midftimes.  findintconvo will return int2 if and only if
  w2 is supplied, and int3 if and only if w3 is supplied.  w1 is 
  required, and int1 will always be returned.  Normally, the weight
  functions should be vectors with the same number of elements as
  midftimes; however, if w1 is empty then the weighting function 
  is taken to be unity.
 
  Note that in order to correctly calculate the convolution, Ca(t) must
  be resampled at evenly spaced time intervals, and this resampled blood
  activity should be passed as Ca\_even.  The times at which it is
  sampled should be passed as ts\_even.  (These can be calculated by
  resampleblood before calling findconvints.)
 
  Then, the convolution of Ca(t) and exp(-k2*t) is resampled at the
  mid-frame times (passed as midftimes) and integrated across frames
  using flengths as dt.
\newpage

%--------------------------------------

{\large\bf RCBF1} {\em a one-compartment (double-weighted integral) rCBF model.}
\begin{verbatim}


         [K1,k2] = rcbf1 (filename, slice)


\end{verbatim}

  A one-compartment rCBF model (without V0 or blood delay and 
  dispersion) implemented as a MATLAB function.  The
  compartmental equation is solved by integrating it across
  the entire study, and then weighting this integral with two
  different weights.  When these two integrals are divided by
  each other, K1 is eliminated, leaving only k2.  A lookup
  table is calculated, relating values of k2 to values of the
  integral.  From this, k2 and be calculated.  From k2, K1 is
  easily found by substitution into the original compartmental
  equation.  See the document "rCBF Analysis Using MATLAB" for
  further details of both the compartmental equations
  themselves, and the method of solution.
\newpage

%--------------------------------------

{\large\bf RCBF2} {\em a two-compartment (triple-weighted integral) rCBF model.}
\begin{verbatim}


        [K1,k2,V0,delta] = rcbf2 (filename, slice)


\end{verbatim}

  rcbf2 implements the three-weighted integral method of calculating
  k2, K1, and V0 (in that order) for a particular slice.  This
  function also returns the delay value calculated for blood
  correction.  It first reads in a great mess of data (viz., the brain
  activity for every frame of the slice, frame start times and
  lengths, plasma activity, and blood sample times).  Then, a simple
  mask is created and used to filter out roughly all points outside
  the head.
  
  The actual calculations follow the procedure outlined in the
  document "RCBF Analysis Using MATLAB".  Occasionally, comments in
  the source code or documentation for various functions involved in
  the analysis will refer to equations in this document.  The most
  relevant functions in this respect are rcbf2 itself, correctblood
  and findintconvos.
  
  The starting point of the three-weighted integration method is Eq.
  10 of the RCBF document.  The left hand side of this equation, rL,
  is calculated for every pixel.  Then, a lookup table relating a
  series of k2 values to the rR (the right-hand side of Eq. 10) is
  calculated.  This lookup table should have a few hundred elements,
  as calculating rR is considerably more expensive than calculating
  rL.  Since rL and rR are equal, we use the pixel-wise values of rL
  to lookup k2 for every pixel.
  
  Then, Eq. XXX is used to calculate K1.  This requires calculating
  the moderately complicated int\_activity (left hand side of Eq. XXX)
  and the extremely complicated k2\_conv\_ints (right hand side).
  However, the expression for k2\_conv\_ints appeared already in the
  numerator of rR, so we preserve that lookup table as conv\_int1 and
  use it to lookup k2\_conv\_ints.  These two long vectors (with one
  number for every pixel) are then divided to get K1.  Finally, V0 is
  calculated via Eq. YYY.
\newpage

%--------------------------------------

{\large\bf DEMO} {\em  Demontrate the EMMA blood analysis package.}
\begin{verbatim}


      demo(slice_number, frame_number)


\end{verbatim}

  Hard-coded to use the arnaud\_20547 data
  file, but allows input of a slice and
  frame to display.
\newpage

%--------------------------------------

{\large\bf RESAMPLEBLOOD} {\em  resample the blood activity in some new time domain}
\begin{verbatim}


   [new_g, new_ts] = resampleblood (handle, type[, samples])


\end{verbatim}

   reads the blood activity and sample timing data from the study
   specified by handle, and resamples the activity data at times
   specified by the string type.  Currently, type can be one of 'even'
   or 'frame'.  For 'even', a new, evenly-spaced set of times will be
   generated and used as the resampling times.  For 'frame', the mid
   frame times will be used.  In either case, the resampled blood
   activity is returned as new\_g, and the times used are returned as
   new\_ts.
 
   The optional argument samples specifies the number of samples
   to take.  If it is not supplied, resampleblood will resample the
   blood data at roughly 0.5 second intervals.
\newpage

%--------------------------------------





\end{document}






