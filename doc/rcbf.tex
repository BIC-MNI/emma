\documentstyle[12pt,fullpage]{article}

\title{rCBF Analysis Using Matlab}
\author{Mark Wolforth and Greg Ward}
\date{Wednesday July 21, 1993}

\begin{document}

\maketitle
\newpage

\tableofcontents

%--------------------------------------------------------------
\newpage
\section{Introduction}


\section{Mathematical Analysis}

\subsection{Introduction}

The two-compartment blood flow problem can be characterized by the
following three equations:

\begin{equation}
\frac{dM}{dt} = K_{1}C_{a}(t) - k_{2}M(t)      \label{eq:2comp1}
\end{equation}
\begin{equation}
M(t) = K_{1}C_{a}(t) \otimes e^{-k_{2}t}       \label{eq:2comp2}
\end{equation}
\begin{equation}
A(t) = M(t) + C_{a}(t)V_{0}                    \label{eq:2comp3}
\end{equation}

In these equations, A(t) is the PET data collected over a set of
frames, $C_{a}(t)$ is the delay and dispersion corrected arterial
blood sample data, and M(t) is the radioactive activity present in
cerebral tissue.  We know both A(t) and $C_{a}(t)$, but cannot know
M(t) without knowing $V_{0}$.  We wish to solve these equations for
$K_{1}$, $k_{2}$, and $V_{0}$.  Of course, one approach would be to
try and perform a curve fitting using these three equations.  However,
this approach would be quite computationally intensive since the
fitting would need to be performed for every pixel of a 128x128 pixel
image.

\subsection{Double-Weighted Integration Method}

\label{sec:double_weight}

Since the time required to perform a curve fitting would be
prohibitive, a simpler approach to solving the problem is required.
This is to use a weighted integration method.  We initially approached
the problem by assuming that $V_{0}$ was negligibly small.  In this
case, A(t) and M(t) become equal, and equation (\ref{eq:2comp3}) is
eliminated.  Taking equation (\ref{eq:2comp2}) and integrating both
sides from 0 to the end of the last frame, we get:

\begin{equation}
\int_{0}^{T} A(t) dt = K_{1} \int_{0}^{T} C_{a}(t) \otimes e^{-k_{2}t} dt
\end{equation}

We can then take this equation, and divide it by a weighted version of
the same function:

\begin{equation}
\frac{\int_{0}^{T} A(t) dt}{\int_{0}^{T} A(t) t dt} = \frac{K_{1} \int_{0}^{T} C_{a}(t) \otimes e^{-k_{2}t} dt}{K_{1} \int_{0}^{T} C_{a}(t) \otimes e^{-k_{2}t} t dt}  \label{eq:1comp}
\end{equation}

$K_{1}$ cancels out of this equation, leaving us with an equation that
only involves $k_{2}$.  The left side of equation (\ref{eq:1comp}) is
easily evaluated by integrating a slice over its frames.  The right
side of equation (\ref{eq:1comp}) is not easily solved for $k_{2}$, so
a different approach was taken.  A look-up table was generated
relating values of $k_{2}$ to resulting values of the right hand side
of equation (\ref{eq:1comp}).  A linear interpolation was then
performed to choose values of $k_{2}$ from this look-up table for each
point in the left hand side of equation (\ref{eq:1comp}).


\subsection{Triple-Weighted Integration Method}

In order to model the complete two-compartment system, we must be able
to solve:

\begin{equation}
A(t) = K_{1}C_{a}(t) \otimes e^{-k_{2}t} + C_{a}(t)V_{0}  \label{eq:full}
\end{equation}

In section \ref{sec:double_weight}, we solved the equation by
multiplying by two different weights and then dividing.  We can take a
similar approach with the full two-compartment equation.  We can
weight equation (\ref{eq:full}) with three different weights and then
integrate:

\begin{equation}
\int_{0}^{T} w_{1}A(t)dt = K_{1} \int_{0}^{T} w_{1}C_{a}(t) \otimes e^{-k_{2}t} dt + V_{0} \int_{0}^{T}w_{1}C_{a}(t) dt  \label{eq:fullweight1}
\end{equation}

\begin{equation}
\int_{0}^{T} w_{2}A(t)dt = K_{1} \int_{0}^{T} w_{2}C_{a}(t) \otimes e^{-k_{2}t} dt + V_{0} \int_{0}^{T}w_{2}C_{a}(t)dt  \label{eq:fullweight2}
\end{equation}

\begin{equation}
\int_{0}^{T} w_{3}A(t)dt = K_{1} \int_{0}^{T} w_{3}C_{a}(t) \otimes e^{-k_{2}t} dt + V_{0} \int_{0}^{T}w_{3}C_{a}(t) dt  \label{eq:fullweight3}
\end{equation}

By multiplying equation (\ref{eq:fullweight1}) by $V_{0} \int_{0}^{T}
w_{3}Ca(t) dt$ and equation (\ref{eq:fullweight3}) by $V_{0}
\int_{0}^{T} w_{1}Ca(t) dt$, and then subtracting the two, we may eliminate the
$V_{0}$ term.  A similar operation can be performed on equation
(\ref{eq:fullweight2}) and equation (\ref{eq:fullweight3}).  This
leaves two equations that do not contain $V_{0}$.  They may then be
divided to produce:

\begin{equation}
\frac{\int_{0}^{T} w_{3}C_{a}(t)dt \cdot \int_{0}^{T} w_{1}A(t)dt - \\
      \int_{0}^{T} w_{1}C_{a}(t)dt \cdot \int_{0}^{T} w_{3}A(t)dt}
     {\int_{0}^{T} w_{3}C_{a}(t)dt \cdot \int_{0}^{T} w_{2}A(t)dt - \\
      \int_{0}^{T} w_{2}C_{a}(t)dt \cdot \int_{0}^{T} w_{3}A(t)dt}
=
\frac{K_{1}\left\{\int_{0}^{T} w_{3}C_{a}(t)dt \cdot \int_{0}^{T} w_{1}C_{a}(t)\otimes e^{-k_{2}t} dt - \\
                  \int_{0}^{T} w_{1}C_{a}(t)dt \cdot \int_{0}^{T} w_{3}C_{a}(t)\otimes e^{-k_{2}t} dt
           \right\}}
     {K_{1}\left\{\int_{0}^{T} w_{3}C_{a}(t)dt \cdot \int_{0}^{T} w_{2}C_{a}(t)\otimes e^{-k_{2}t} dt - \\
                  \int_{0}^{T} w_{2}C_{a}(t)dt \cdot \int_{0}^{T} w_{3}C_{a}(t)\otimes e^{-k_{2}t} dt
           \right\}}
\end{equation}



\subsection{Delay and Dispersion Correction}


\end{document}






