\documentstyle[11pt,fullpage]{article}

\title{Data Simulation and Program Verification Using MATLAB and Maple V}
\author{Greg Ward \and Mark Wolforth}
\date{9 September, 1993}

\def\code#1{{\tt #1}}

% \def\simuldir{~greg/matlab/testing}
\newcommand{\units}[1]{\mbox{$\rm{#1}$}}
\newcommand{\funits}[2]{\mbox{$\rm\frac{#1}{#2}$}}

\begin{document}

\maketitle

\tableofcontents

%--------------------------------------------------------------
% \newpage
\section{Introduction}

This document is meant to provide a basic guide to performing data
simulation (and subsequent program verification) using Maple V and
MATLAB.  It was developed in conjunction with the rCBF analysis
package and its documentation, and uses the simulation of rCBF data
and verification of the analysis package as illustrative tools.
Furthermore, much of the information is specific to the Silicon
Graphics workstation environment at the McConnell Brain Imaging
Centre, particularly the presence of current versions of Maple,
MATLAB, and the rCBF analysis programs for MATLAB.

%--------------------------------------------------------------
% \newpage
\section{Software Tools}

Maple and MATLAB are both commercially available mathematical software
packages with rather different focuses: Maple deals with mathematics
almost entirely symbolically, while MATLAB is entirely numeric.  Maple
is more concerned with traditional algebraic entities like expressions
and functions, whereas MATLAB's primary data structure is the matrix
(and the special cases of vectors and scalars).  Thus, MATLAB is
better suited to dealing with scientific data, and Maple is
well-adapted to dealing with the theoretical functions which model the
data.

EMMA and rCBF are both sets of tools written at the MNI for MATLAB,
largely using MATLAB's built-in language.  EMMA consists of functions
to read both image and ancillary data from MINC files; rCBF uses EMMA
to read this data, and MATLAB's collection of mathematical tools (with
some enhancements) to calculate the kinetic parameters of either
the one- or two-compartment model of regional cerebral blood flow.
See the documents ``EMMA User Manual'' and ``rCBF Analysis Using
MATLAB'' for more information.

\subsection{A Brief Introduction to Maple V}

Since Maple V is not currently a heavily-used tool at McBIC, we will
first present a brief introduction to its capabilities.

\subsubsection{Starting Maple}

There are currently two interfaces to Maple accessible on the SGI
workstations at McBIC.  Typing \code{maple} at the shell prompt starts
up a {\em very} basic TTY-style interface to Maple.  This interface
provides minimal command-line editing, no command-line history, and
poor access to graphics.  (The easiest method is to use
\code{xterm(1)} on an X Windows display---either an SGI console or Sun
X terminal will work---and activate the Tektronix window {\em before}
issuing the Maple plot command.)

Alternatively, you can type \code{xmaple} at the shell prompt, as long
as you are using an X Windows display.  This opens up a new window for
Maple's use.  The interface is essentially the same as the TTY version,
but here you can at least edit command lines, recall previous command
lines (albeit in a genuinely bizarre way that must be experienced to be
believed), and put plots into their own windows automatically.  Neither
interface to Maple is really satisfactory, unless you make no typing
errors and have no objections to retyping nearly identical command
lines.

\subsubsection{Creating and Assigning Expressions}

Creating mathematical expression in Maple is quite easy.  For example,
try typing---and the semi-colon {\em is} important---
\begin{verbatim}
     sin (x) / x^2;
\end{verbatim}
and see that Maple ``prettyprints'' the expression in a form somewhat
more legible than the standard computer-readable format.  If we now wish
to manipulate this function in some way without retyping it, it is
necessary to assign it to a variable.  Actually, this is not strictly
true: you can access the most recent result of a Maple statement with
the \code{"} symbol, the second most recent result with \code{""}, and
the third most recent with \code{"""}.  In fact, we will use this
method to call the above expression \code{f}:
\begin{verbatim}
     f := ";
\end{verbatim}
%
Note the use of \code{:=} for assignment.  If you had typed
\begin{verbatim}
     f = ";
\end{verbatim}
then Maple would have simply created an ``equation'' object, which
would be accessible only through \code{"}.  In the former case,
\code{f} is logically equivalent to \verb|sin (x) / x^2|; that is,
using \code{f} in an expression is the same as using \verb|sin (x) /
x^2|.  In the second case, \code{f} is only mathematically equivalent,
and only within the context of the expression itself. {\em This}
expression could be passed to a Maple function such as \code{solve} to
find the inverse of the expression to which \code{f} is equated.
(Don't bother trying it with the current expression---it's much too
complicated for Maple to solve symbolically for
\code{x}.  However, if you type \code{y~=~sin (x);} and then \code{solve (",~
x);}, Maple should solve correctly for \code{x}.  Or, to do the same thing in a
single step, try \code{solve (y = sin (x), x);}.)

Remember that you must always end a Maple statement with a semicolon
or a colon.  The only difference between the two is that using a colon
will cause Maple to suppress the output of the command.  This can be
handy in scripts (see below) when you do not want to mess up the
display with intermediate steps.

If you now want to make sure that f is in fact {\em logically} equivalent
to \verb|sin (x) / x^2|, just type \code{f;} and Maple will report on
the ``value'' of \code{f}.

Now, as an example of something Maple {\em can} do with \code{f}, type
\begin{verbatim}
     fsolve (f, x);
\end{verbatim}
The \verb|fsolve| function, as opposed to \verb|solve|, solves
equations numerically and thus will be able to find a solution more
often than \verb|solve|.  Note that \verb|fsolve| expects an equation
as its first argument, and we have passed a simple expression with no
\verb|=| anywhere in it.  In this case, \verb|fsolve| interprets
\verb|f| as the equation \verb|f = 0|---thus, we have just computed a
zero of $\frac{sin(x)}{x^{2}}$.

\subsubsection{Using Maple's Built-in Functions}

Some other built-in functions that will come in very handy are
\verb|subs|, \verb|evalf|, \verb|int|, and \verb|simplify|.
\verb|subs| allows you to substitute a portion of one expression 
with another expression.  For example, try entering
\begin{verbatim}
     subs (x=y, f);
\end{verbatim}
and observe that Maple replaces \verb|x| in \verb|sin (x) / x^2| with \verb|y|.
(Note that this does {\em not} change the value of \verb|f|.  You can
generally only do that with another assignment.)  We can combine
\verb|evalf| and \verb|subs| to evaluate the expression at certain
values of \verb|x|.  For instance, 
\begin{verbatim}
     evalf (subs (x=Pi, f));
\end{verbatim}
assures us that $\pi$ is indeed a zero of $\frac{sin(x)}{x^{2}}$, at
least to the precision Maple is capable of. 

Next, let's symbolically integrate \verb|f| with respect to x:
\begin{verbatim}
     int (f, x);
\end{verbatim}
(That peculiar looking expression \verb|Ci| in the output is one of
Maple's built-in mathematical functions.  Since $\frac{sin(x)}{x^{2}}$
is not easily integrated, Maple uses \verb|Ci|, which involves the
gamma function and some other non-elementary functions, to express
this integral.  Type \verb|? Ci| for more information, including a
list of all of Maple's built-in mathematical functions).  Alternately,
we can compute a definite integral:
\begin{verbatim}
     int (f, x=Pi/2..Pi);
\end{verbatim}
We can find out what this is as a floating point number using
\verb|evalf| again:
\begin{verbatim}
     evalf (");.
\end{verbatim}

\subsubsection{Control Structures}

Maple provides a number of traditional programming structures,
including \verb|for/while/do| looping and \verb|if/then/else|
statements.  For detailed help, type \verb|? for| or \verb|? if|;
since we will be using the \verb|for| loop extensively in our data
simulation, a brief example is provided here.

A simple example using \verb|for| is:
\begin{verbatim}
     for i from 1 to 10 do
        lprint (i);
        lprint (evalf (sin (i)));
     od;
\end{verbatim}
which simply prints the integers from 1 to 10 and the sine of each
integer.  Although \verb|for| is generally used within programs, you
can type the above example into Maple interactively.


\subsubsection{Outputting Data}

Maple provides several built-in functions for saving expressions
(which can, of course, be either symbolic expressions or the numerical
output of \verb|evalf| or \verb|fsolve|) to text files.
Unfortunately, none of them work.  The option that appears most
convenient is the \verb|open| and \verb|writeln| pair.
Hypothetically, one could enter:
\begin{verbatim}
     readlib (write);                     # to have access to open/writeln
     open (foobar);                       # prepare to write to file foobar
     writeln (`Hello There`);             # and write to it
     writeln (sin (x));
     writeln (int (sin (x), x=0..Pi));
     writeln (evalf (subs (x=Pi/2, sin(x))));
\end{verbatim}
to write a string, a simple numerical expression, and two numbers (2
and 1) to the text file \verb|foobar|.  Unfortunately---as of this
writing, at least---Maple's \verb|open| function does not work: it 
appears to do nothing, and the Maple prompt never returns.

Since it is (apparently) impossible to coax Maple into directly
writing to a text file, we will make use of the input/output
redirection provided by the Unix shell.  To do this, we first write a
text file containing the list of Maple commands whose output we wish
to capture.  This file can be terminated by a Maple ``\verb|quit;|''
command if we desire, but this is not necessary: when Maple is
finished reading from the file, it then switches to reading from your
terminal, where you can continue the session or type ``\verb|quit;|'' to
end it.  For instance, create a file called \verb|makesin| in your
current directory with the following contents:
\begin{verbatim}
     f := sin (x):
     for i from 0 by 0.5 to 10 do
        lprint (evalf (subs (x=i, f)));
     od;
\end{verbatim}
From the Unix shell prompt, type
\begin{verbatim}
     maple < makesin
\end{verbatim}
and note that we get $sin(x)$ evaluated at $x$ = 0, 0.5, 1, \ldots, 10.  To
save that output to a file called \verb|sinoutput|, we would have
typed \verb|maple < makesin > sinoutput|.  If we wish to use these
numbers in any way, the output file will most likely have to be edited
to remove the Maple prompts and commands read from \verb|makesin|.


%\subsection{A Very Brief Introduction to MATLAB, EMMA, and rCBF}

%Please consult the documents ``EMMA User Manual'' and ``rCBF Analysis
%Using MATLAB'' if you are not yet conversant with MATLAB and the EMMA
%and rCBF sets of functions.



%--------------------------------------------------------------
% \newpage
\section{Data Simulation}

\subsection{Mathematical Background}
For a more detailed mathematical exposition, please consult ``rCBF
Analysis Using MATLAB'' or one of its references.  Briefly, we wish
to generate a set of plasma data; pick reasonable values of $K_{1}$,
$k_{2}$, and $V_{0}$; and generate brain activity data $A^{*}(t)$ according
to the model equation
\begin{equation}
A^{*}(t) = K_{1} (C_{a}(t) \otimes e^{-k_{2}t}) + C_{a}(t)V_{0}.
\label{eq:pet_model}
\end{equation}
Note, however, that PET data is in reality
\begin{equation}
A(t_{mid,i}) = \frac{\int_{t_{start,i}}^{t_{end,i}} A^{*}(t) \,dt}{t_{end,i} - t_{start,i}}
\label{eq:frameint}
\end{equation}
where $t_{start,i}$ and $t_{end,i}$ are the start and end times of
frame $i$, and $t_{mid,i}$ is the mid-frame time ($ = (t_{start,i} +
t_{end,i}) / 2$).  Thus, in order to model real data, we must integrate
both sides of Eq. \ref{eq:pet_model} across each frame to obtain
\begin{equation}
A(t_{i}) = \frac{\int_{t_{start,i}}^{t_{end,i}} [K_{1} (C_{a}(t) \otimes e^{-k_{2}t}) + C_{a}(t)V_{0}] \,dt}{t_{end,i} - t_{start,i}}.
\label{eq:pet_data_model}
\end{equation}

Also, it has been found that the formula
\begin{equation}
C_{a}(t) = \sum_{i=1}^{4} a_{i} t^{n_{i}} e^{-t/b_{i}}
\label{eq:gen_ca}
\end{equation}
creates realistic-looking simulated plasma data when $a_{i}$ = (0.055,
10, 0.7, 0.2), $n_{i}$ = (6, 1, 1, 1), and $b_{i}$ = (1.8, 22, 180, 300)
[REF!!!].  The implementation of this formula will thus be the first step of
our simulation.  Also, the ``data'' created with this formula is ideal
plasma data, in that it is subject to neither dispersion or delay.
Thus, we can skip the blood correction step when analysing the
simulated data.

\subsection{Generating the Simulated Data in Maple}

Most of the Maple commands described in this section are also
available in various files in the directory
\verb|/local/matlab/toolbox/local/simul|.  In the remainder of this
document, it will be assumed that you have created a symbolic link to
this directory using the shell command
\begin{verbatim}
     ln -s /local/matlab/toolbox/local/simul simul
\end{verbatim}
from one of your own directories.  That way, you can refer to files in
the simulation directory using ``\verb|simul/filename|'' rather than
the unwieldy ``\verb|/local/matlab/toolbox/local/simul/filename|''.
Thus, any time that this document refers to the directory
\verb|simul/|, we are referring to \\
\verb|/local/matlab/toolbox/local/simul|.

The commands are stored as text files which are read into Maple with
the \verb|read| command.  For example, to read the file
\verb|makeplasma| (assuming you are in the directory where you created
the symbolic link \verb|simul|), enter 
\begin{verbatim}
     read `simul/makeplasma`;
\end{verbatim}
in Maple.  Note the use of backquotes \verb|`| to denote a string in Maple.

There are a few minor differences between the commands presented here
and those available on-line; particularly, most commands in the
on-line version are terminated with colons rather than semicolons to
suppress Maple's output.  Also, none of the unit conversions described
herein are done by the scripts \verb|makeplasma|, \verb|makepet|, or
\verb|intframes|.  The reason for this is that there is also a script 
\verb|makeall| which calls the three ``workhorse'' scripts.  
\verb|makeall| outputs all required data (the calculated plasma 
and brain activity, and the semi-arbitrarily chosen frame times and
lengths) in the appropriate units, as well as performing the unit
conversions necessary to the next step in the process.  After you have
worked through and understand the example code presented here, it
would be best to run the automated version to generate the output.

\subsubsection{Units}

In order to avoid headaches, we will do all calculations to generate
the simulated data using units of \units{counts}, \units{mL_{blood}},
\units{g_{tissue}}, and \units{sec}.  This will necessitate converting
the input \verb|Ca| from \units{\frac{counts}{g_{blood}\:sec}} to
\funits{counts}{mL_{blood}\:sec}, and the output \verb|A| from
\funits{counts}{g_{tissue}\:sec} to
\funits{nCi}{mL_{tissue}}, the standard units for PET data.
Furthermore, we must be sure to supply values of $K_{1}$, $k_{2}$, and
$V_{0}$ that make sense in units of
\funits{mL_{blood}}{g_{tissue}\:sec}, \units{sec^{-1}}, and
\funits{mL_{blood}}{g_{tissue}}.


\subsubsection{Generating Plasma Activity}

First, we wish to generate $C_{a}(t)$ according to Eq. \ref{eq:gen_ca}.  The
following Maple statements create 4-element vectors of $a_{i}$, $n_{i}$, and
$b_{i}$, and perform the summation:

\begin{verbatim}
     a := array (1..4, [0.055, 10, 0.7, 0.2]);
     n := array (1..4, [6, 1, 1, 1]);
     b := array (1..4, [1.8, 22, 180, 300]);

     Ca := 0;
     for i from 1 to 4 do
        Ca := Ca + 500 * a[i] * t ^ (n[i]) * exp (-t / (b[i]));
     od;
\end{verbatim}

Note the scaling of \verb|Ca| by a factor of 500.  This is simply to
get values of the appropriate order of magnitude, namely a function
peaking at around 150,000.  Since this is comparable to what is
experimentally observed in rCBF studies (neglecting dispersion
correction, which results in a slightly higher peak), we shall assign to the
generated \verb|Ca| units of \funits{counts}{g_{blood}\:sec},
which are the units that experimental plasma data is recorded in---and
therefore what the rCBF programs expect it to be in.

Since we will have to multiply \verb|Ca| by a scale factor in order to
create $A(t)$ in the expected units, it would be wise to save
\verb|Ca| to a file now, while it is in the desired units.  As
discussed above, however, Maple does {\em not} make this terribly
easy.  Thus, there is a Maple script in
\verb|simul/| called \verb|makeall| which
automates data generation; this file can be redirected to Maple's
input, and the resulting output redirected to a file which can then be
manually edited and split up for converting to MINC files.  This
procedure will be discussed below.

\subsubsection{Generating Brain Activity}

The first step in calculating $A(t)$ is to convert \verb|Ca| to 
\funits{counts}{mL_{blood}\:sec}.  Since $1.05 \units{g} = 1
\units{mL}$ is assumed for both blood and tissue, this is done by
simply multiplying by 1.05:
\begin{verbatim}
     Ca := Ca * 1.05:
\end{verbatim}
Now, we wish to pick values of $K_{1}$, $k_{2}$, and $V_{0}$.  Note
that the ratio $\frac{K_{1}}{k_{2}}$ should be between 0.6
and 1.0 \funits{mL_{blood}}{g_{tissue}}; $V_{0}$ should be around 0.01 to
0.05 \funits{mL_{blood}}{g_{tissue}}.  With this in mind, we pick the
following parameters:
\begin{eqnarray*}
K_{1} &=& 75 \: \funits{mL_{blood}}{100\:g_{tissue}\:{min}} \\
k_{2} &=& 1 \: \units{min^{-1}} \\
V_{0} &=& 4 \: \funits{mL_{blood}}{100\:g_{tissue}}
\end{eqnarray*}
and enter them into Maple, with conversions to the desired units:
\begin{verbatim}
     K1 := 75 / 60 / 100;
     k2 := 1 / 60;
     V0 := 4 / 100;
\end{verbatim}
Of course, you should try a fairly wide range of parameter values,
even if they don't necessarily make physiological sense: as long as
$k_{2}$ is not out of the range of the main lookup tables, the MATLAB
rCBF programs should be able to recover all three values fairly well.
These assignment statements are in the file \verb|makepet|; you should
copy this file to your directory and edit it to reflect whatever
parameter values you wish to use.

Now we will perform the steps of Eq. \ref{eq:pet_model} to find
$A^{*}(t)$.  First, recall that the convolution operator~$\otimes$
denotes the following operation:
\begin{equation}
f(t) \otimes g(t) \equiv \int_{0}^{t} f(\tau) g(t-\tau) \: d\tau.
\label{eq:convolution} 
\end{equation}
In Maple, we perform this operation with the following statements:
\begin{verbatim}
     arg1 := subs (t=tau, Ca);
     arg2 := exp (-k2 * (t-tau));
     convo := int (arg1 * arg2, tau=0..t);
\end{verbatim}
(\verb|arg1| and \verb|arg2| are used simply for clarity; the whole
operation is performed in one line in \verb|makepet|)

Unfortunately, Maple is unable to symbolically integrate the entire
expression.  We will still be able to numerically evaluate it, but we
cannot use Maple to generate a plot of the function (or rather, the
eventual end product $A^{*}(t)$) to ensure that it ``looks right'' for PET
brain activity.

Now, we complete the calculations of Eq. \ref{eq:pet_model}, and find
$A^{*}(t)$:
\begin{verbatim}
     Astar := K1 * convo + V0 * Ca;
\end{verbatim}

\subsubsection{From Theoretical Functions to Simulated Data}

To convert $A^{*}(t)$ to $A(t)$, we must create a set of frames across
which to integrate $A^{*}(t)$, and then numerically integrate
\verb|Astar| across each one.  Unfortunately, because of the $\int
\tau^{6} e^{(-\tau)} d\tau$ in \verb|Astar|, Maple is unable to integrate
the entire expression.  It can {\em evaluate} \verb|Astar| at a given
value of t---that is, 
\begin{verbatim}
     evalf (subs (t=5, Astar));
\end{verbatim}
successfully calculates $A^{*} (5)$---but it cannot integrate
\verb|Astar|, even numerically.  That is,
\begin{verbatim}
     evalf (int (Astar, t=0..5));
\end{verbatim}
fails to calculate $\int_{t=0}^{5} A^{*} (t) dt$.

Note, however, that the ``frame-by-frame integration'' in
Eq.~\ref{eq:frameint} simply averages the function $A^{*}(t)$ across
each frame.  Thus, as long as the function is nearly linear over each
frame, then simply evaluating \verb|Astar| at each mid-frame time will
be a good approximation to integrating across the frame and dividing
by the frame length.  (In fact, this is the method used to perform
Eq.~\ref{eq:frameint} in previous implementations of the rCBF
analysis, where linear interpolation between data points was used to
``evaluate'' the function.)

We will pick the frame start times and frame lengths (all in seconds)
as follows (note that splitting the input across lines poses no
problems, because Maple only looks for the terminating semicolon):
\begin{verbatim}
     FrameStarts := array (1..21, 
         [0,5,10,15,20,25,30,35,40,45,50,55,60,70,80,90,100,110,120,140,160]);
     FrameLengths = array (1..21, 
         [4,5, 5, 4, 5, 4, 4, 5, 4, 5, 4, 5, 9,10, 9,10, 10, 10, 19, 19, 20]);
\end{verbatim}
(Actually, these are taken from study \verb|ARNAUD_20547| with the
decimal portion of the frame start times truncated.  Again, feel free
to copy the Maple script---this code is in \verb|intframes|---and
modify it.  Note that gaps between frames are allowed, since they do
occur in real data; overlapping frames, however, are not handled by
the rCBF software.)

Before creating an array of values of $A(t)$, we will scale
\verb|Astar| in order to convert from its present units of
\funits{counts}{g_{tissue}\:sec} to \funits{nCi}{mL_{tissue}}.  This
is done by simply multiplying by the cross-correlation coefficient,
which relates blood-count data to brain activity data.  This
coefficient is generally taken to be 0.11 \funits{nCi / mL}{count / g
\: sec}, therefore:
\begin{verbatim}
    Astar := Astar * 0.11;
\end{verbatim}

To create 21 values of $A(t)$ to simulate actual PET data, then, the
following \verb|for| loop is used:
\begin{verbatim}
     for i from 1 to 21 do
         MidTime := FrameStarts [i] + (FrameLengths [i] / 2)
         A [i] := evalf (subs (t=MidTime, Astar));
     od
\end{verbatim}

\subsection{Saving the Simulated Data in MINC Files}

At this stage, you should have the Maple scripts \verb|makeplasma|,
\verb|makepet|, \verb|intframes|, and \verb|makeall| in your own
directory.  Also, if you wish to make any modifications to the scripts
(e.g., to change the values of $K_1$, $k_2$, or $V_0$, or the frame
times or lengths) you should do so now.

Since Maple's outputting facilities are non-functional, we will have
to make use of some of the tools provided by Unix to make the job of
getting sensible numerical output out of Maple easier.  The first is
to use Maple as a pipe, reading its input from the file \verb|makeall|
(instead of from a terminal) and writing output to some other file.
Then, we will use a simple \verb|awk(1)| script to strip out
unnecessary output and split the output into several files.

Finally, we will convert the processed Maple output (which are
human-readable numbers) to machine-readable ``raw'' format, and then
to a MINC file.

\subsubsection{Getting Numbers Out of Maple}

To run the \verb|makeall| script and send the output to a file called
\verb|output|, type the following from the shell prompt:
\begin{verbatim}
     maple < makeall > output 
\end{verbatim}
Since \verb|makeall| does not end with a \verb|quit| command, Maple
will simply wait for input from the terminal.  Just type \verb|quit;|
and Maple will exit.

The output file now contains all the data we need, as well as Maple's
prompts, input from the script, and occasional Maple status reports.
To get rid of the unneeded text and split the file into several
smaller files---one for each of \verb|Ca|, \verb|A|, \verb|FrameTimes|,
and \verb|FrameLengths|, as well as for the times at which \verb|Ca|
was sampled---we will use \verb|parse-output.awk| (also in
\verb|simul/|).  The command
\begin{verbatim}
     awk -f parse-output.awk output
\end{verbatim}
will scan the file \verb|output| for lines starting with
``\verb|START|'' followed by a filename; write all lines that are
numeric data (i.e. start with a digit) to the file named on the
\verb|START| line; and stop writing to that file when a line starting
with ``\verb|END|'' is encountered.  This is repeated for as many
\verb|START-END| pairs are found in \verb|output|.  If you have not
modified either \verb|makeall| or \verb|output| from the copies in
\verb|simul/|, then \verb|output| will contain
\verb|START-END| pairs to create the following files:
\begin{description}
\item \verb|plasma.txt| - the simulated plasma activity data.
\item \verb|sample-times.txt| - the start times of simulated ``blood samples''
\item \verb|sample-lengths.txt| - the length of simulated blood samples.  
Together, \verb|sample-times.txt| and \verb|sample-lengths.txt|
specify the mid-sample times, which are the times at which \verb|Ca|
was sampled to create \verb|plasma.txt|.  However, we must store the
sample start times and lengths because they are what the rCBF software
expects to find in the MINC file.
%This is just
%the Maple expression \verb|Ca| evaluated at {\em mid-\/}sample times; i.e.,
%if the sample start times are 0, 2, 4, \ldots, 180 sec and all sample lengths 
%are 2 sec, then the mid sample times will be 1, 3, \ldots, 181 sec.  
\item \verb|frame-times.txt| - the start times of every frame of the
simulated PET study.
\item \verb|frame-lengths.txt| - the lengths of every frame in the
simulated study.
\item \verb|pet.txt| - the simulated PET data, sampled at the
mid-frame times (which are specified by the frame timing data in
\verb|frame-times.txt| and \verb|frame-lengths.txt|).
\end{description}

\subsubsection{Putting Text Data Into a MINC File}

While it is quite straightforward to read simple text files contain
columns of numbers into MATLAB (using the \verb|load| command), it
will be far more convenient to put the data into a MINC file as
expected by the rCBF programs.

This is done by means of a number of utility programs provided with
the EMMA package: \verb|micreate|, \verb|micreateimage|,
\verb|micreatedim|, and \verb|micreatevar|.  (Note: in order to
improve conformance to the MINC standard and simplify certain
operations within MATLAB, \verb|micreate| and \verb|micreateimage|
will eventually be merged into one program.  As of this writing,
however, they are distinct.)  Since these programs are really intended
to be run by other programs rather than by humans, they have rather
inflexible command line syntaxes: generally speaking, all arguments
must be provided in the exact order specified by the program's help.
The help is obtained by simply running each program with no arguments.
If you type the commands shown here exactly as is, you should have no
problems.

First, we must create the MINC file.  Type the following at the shell
prompt:
\begin{verbatim}
     micreate - simul.mnc
\end{verbatim}
The first argument to \verb|micreate| is a ``parent file,'' which is
generally used when creating MINC files of results calculated from a
particular study---this allows the new file to inherit attributes such
as the patient name or slice spacing.  Supplying a single \verb|-|
tells \verb|micreate| there is no such parent file.

Next, we create the image dimensions and variable.  Since we have
created only one brain activity curve, the image will be 1 by 1, with
1 slice and 21 frames.  The following command will create such an
image variable:
\begin{verbatim}
     micreateimage simul.mnc 21 1 1 1
\end{verbatim}

Now, we create the dimension and variables required to store the
plasma activity data.  There is only one dimension involved, and its
length is the number of simulated blood samples taken, or the number
of lines in the file \verb|plasma.txt|.  If \verb|makeall| was not
modified, this will be 91.  Thus, to create a NetCDF dimension
\verb|sample| with a length of 91, enter
\begin{verbatim}
     micreatedim simul.mnc sample 91
\end{verbatim}
If you {\em have} changed \verb|makeall|, you may need to re-count the
number of lines in \verb|plasma.txt|.  Do this by running the command
\begin{verbatim}
     wc plasma.txt
\end{verbatim}
The first number given is the number of lines.  If applicable, use
this number for the dimension length (in place of 91) in the above
call to \verb|micreatedim|.

Now, we must create several variables for the blood data.  The
variable \verb|blood_analysis| only exists as a tag to indicate that
the blood data is present in the MINC file.  (Actually, it is also
used to set up a hierarchy of MINC variables, but this aspect of MINC
files is ignored by this simulation.) The variables
\verb|sample_start|, \verb|sample_length|, and
\verb|corrected_activity| will contain useful data.  To create these
variables, type:
\begin{verbatim}
     micreatevar simul.mnc blood_analysis NC_LONG 0 -
     micreatevar simul.mnc sample_start NC_DOUBLE 1 sample
     micreatevar simul.mnc sample_length NC_DOUBLE 1 sample
     micreatevar simul.mnc corrected_activity NC_DOUBLE 1 sample
\end{verbatim}

Finally, we must convert the human-readable text files to
machine-readable ``raw'' format, and then put the data into the
newly-created MINC file.  There is a little conversion utility called
\verb|atod| (for ASCII to double) which we will run on our .txt files
as follows:
\begin{verbatim}
     simul/atod < plasma.txt > plasma.dat
     simul/atod < sample-times.txt > sample-times.dat
     simul/atod < sample-lengths.txt > sample-lengths.dat
     simul/atod < pet.txt > pet.dat
     simul/atod < frame-times.txt > frame-lengths.dat
     simul/atod < frame-lengths.txt > frame-lengths.dat
\end{verbatim}
Again, if you have changed any of the filenames, these commands should
be adapted appropriately.

We first write the raw file of image data to all 21 frames of the MINC
file.  Note that \verb|miwriteimages| expects slice and frame numbers
to be zero-based, so that frame numbers run from 0 to 20:
\begin{verbatim}
     miwriteimages simul.mnc - 0,1,2,3,4,5,6,7,8,9,10,11,12,13,14,15,16,17,18,19,20 pet.dat
\end{verbatim}
Also, do {\em not} include spaces in the list of frame numbers.

Writing the other variables is done as follows (again, be sure to
change 91 or 21 to the number of samples or frames if you have changed
\verb|makeall|):
\begin{verbatim}
     miwritevar simul.mnc sample_start 0 91 sample-times.dat
     miwritevar simul.mnc sample_length 0 91 sample-lengths.dat
     miwritevar simul.mnc corrected_activity 0 91 plasma.dat
     miwritevar simul.mnc time 0 21 frame-times.dat
     miwritevar simul.mnc time-width 0 21 frame-lengths.dat
\end{verbatim}

\subsection{Running rCBF on the Simulated Data}

\verb|[K1,k2,V0] = rcbf2 ('simul.mnc', 1, 1, 0)|.

\end{document}
